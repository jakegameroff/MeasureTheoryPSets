%! TeX root: ../main.tex
\noindent \textbf{5.2.} Let \( (f_n)_{n \in \mathbb{N} }  \), \( f_n : A \to \mathbb{R} \) be a sequence of measurable functions defined on a measurable set \( A \subseteq \mathbb{R}^{d}  \) such that \( (f_n)_{n \in \mathbb{N} }  \) converges pointwise in \( A \) to a function \( f : A \to \mathbb{R} \). Use Egorov's theorem to show that the set \( A \) can be written as the countable union of measurable sets \( (A_{k} )_{k \in \mathbb{N} _ \geq 0}  \) such that \( m(A_0) = 0 \) and for every \( k \geq 1 \), \( (f_n)_{n \in \mathbb{N} }  \) converges uniformly to \( f \) in \( A_k \).
\begin{proof}
We may first assume that \( A \) is a bounded set. By Egorov's theorem, for each \( k \in \mathbb{N}  \) there exists a closed set \( A_k \subseteq A \) such that \( m(A \setminus A_k) < 1/k \) and \( f_n\) converges uniformly to \( f \) in \( A_k \). Define \( A_0 \coloneqq \bigcap_{k =1 }^{\infty} (A \setminus A_k) \) so that \( A = \bigcup_{k =0}^{\infty} A_k \). Indeed, if \( x \in A \) then either \( \exists \ k \in \mathbb{N} : x \in A_k  \implies x \in \bigcup_{k=0}^{\infty} A_k\) or there is no such \( k \); that is, \( \forall k \in \mathbb{N} : x \notin A_k \implies x \in A \setminus A_k \) for each \( k \in \mathbb{N} \implies x \in \bigcap_{k=1}^{\infty} (A \setminus A_k) = A_0 \implies x \in \bigcup_{k=0}^{\infty} A_k \). On the other hand, if \( x \in A_0 \implies x \in \bigcap_{k=1}^{\infty} (A\setminus A_k) \subseteq A \setminus A_1 \subseteq A \) and if \( x \in \bigcup_{k=1}^{\infty} A_k \implies \exists \ m \in \mathbb{N} : x \in A_m \subseteq A  \).

To see why \( m(A_0) = 0 \), let \( \varepsilon > 0 \) and choose \( N \in \mathbb{N}  \) large enough so that \( 1 / N < \varepsilon  \). Then \( A_0 \subseteq A \setminus A_N \) so that by monotonicity and the construction of \( A_N \), \( m(A_0) \leq m(A \setminus A_N) < \frac{1}{N} < \varepsilon  \). Since \( \varepsilon  \) was arbitrary, we conclude that \( m(A_0) = 0 \). Hence, \( A = \bigcup_{k=0}^{\infty} A_k \) is a countable union with \( m(A_0) = 0 \) and for \( k \geq 1 \), \( f_k \) converges uniformly to \( f \) in \( A_k \).

Now suppose \( A \) is unbounded. We write \( A = \bigcup_{m=1}^{\infty} E_m,\) where \(E_m \coloneqq  A \cap [-m,m]^{d}  \) for each \( m \in \mathbb{N} \). Then each \( E_m \) is measurable (finite intersection of measurable sets), bounded (\( E_m\subseteq [-m,m]^{d}  \)), and \( f_n|_{E_m}  \) is measurable (by lecture) and converges pointwise to \( f \) in \( E_m = A \cap E_m \subseteq A\). Thus, since the claim has been proven when the functions' domains are bounded, we can write \( E_m = \bigcup_{j=0}^{\infty} A_{m, j}  \) where \( A_{m,j}  \) is measurable for each \( j \geq 0 \), \( m(A_{m,0} ) = 0 \), and \( j \geq 1 \implies  \) \( f_n|_{E_m} \) converges uniformly to \( f \) in \( A_{m,j}  \). It follows that
\begin{align*}
	A &= \bigcup_{m=1}^{\infty} \bigcup_{j=0}^{\infty} A_{m,j} = \bigcup_{m=1}^{\infty} \bigcup_{j=1}^{\infty} A_{m,j} \cup \bigcup_{m=1}^{\infty} A_{m,0}. 
\end{align*}
Now let \( A_0 \coloneqq \bigcup_{m=1}^{\infty} A_{m,0}  \). By sub-additivity, \( m(A_0) = m(\bigcup_{m=1}^{\infty} A_{m,0} ) \leq \sum_{m=1}^{\infty}m(A_{m,0} ) = 0\) since \( m(A_{m,0}) = 0  \) for \( m \in \mathbb{N}  \). For each \( m,j \in \mathbb{N}  \), we have that \( f_n |_{E_m} \) converges uniformly to \( f \) in \( A_{m,j}  \) so that \( f_n \) converges uniformly to \( f \) in \( A_{m,j} \cap E_m  = A_{m,j} \). Since \( \bigcup_{m=1}^{\infty} \bigcup_{j=1}^{\infty} A_{m,j}  \) is a countable union, we can rewrite it as \( \bigcup_{k=1}^{\infty} A_k \), a countable union of measurable sets so that for \( k \in \mathbb{N}  \), \( f_n \) converges uniformly to \( f \) in \( A_k \).

Thus, \( A = \bigcup_{k=0}^{\infty} A_k\) has been written as a countable union of measurable sets (since \( k \geq 1 \implies A_k \) is measurable by construction and sets of outer measure zero are measurable); moreover, \( m(A_0) = 0 \) and for each \( k \geq 1 \) we have \( f_n \to f \) uniformly in \( A_k \) as required. Thus, the proof is complete.
\end{proof}
%
