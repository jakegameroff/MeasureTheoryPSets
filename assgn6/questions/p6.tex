%! TeX root: ../main.tex
\noindent \textbf{Problem 6.} For each \( \lambda \in (-\infty,1) \), find \[\lim_{{k} \to {\infty}} \int_{(0,k)} \left(1 - \frac{x}{k} \right)^{k}e^{\lambda x } . \]
\noindent \textbf{Proof.} \\
Fix \( \lambda \in (-\infty,1) \). We first note that the sequence \( (1- (x/k))^{k}  \) converges to \( e^{-x}  \) as \( k \to \infty \) and is monotonically increasing (cf. Lemma 6.1). Thus, for each fixed \( k \in \mathbb{N}  \) and \( x \in (0,\infty) \), \[|f_{k}(x)| = \left ( {1-\frac{x}{k} } \right ) ^{k} e^{\lambda x} < e^{-x}e^{\lambda x} = e^{x (\lambda - 1)}.      \] We must show that \( e^{x(\lambda - 1)}  \) is integrable in order to apply the dominated convergence theorem. To do so, we calculate \[\int_{(0,\infty)} e^{x(\lambda - 1)} = \lim_{{t} \to {\infty}} \int_{(0,t)}e^{x(\lambda - 1)} = \lim_{{t} \to {\infty}} \int_{[0,t]} e^{x(\lambda - 1)}, \] since \( \{ 0,t \}  \) is finite and hence of measure 0, by lecture we have \[ \int_{(0,t)} e^{x(\lambda -1)} = \int_{(0,t)}e^{x(\lambda -1)} + \underbrace{\int_{\{0,t\}}  e^{x(\lambda - 1)} }_{=0} .\] Thus, by lecture, we can take the Reimann integral to find the Lebesgue integral \( \int_{(0,t)} e^{x(\lambda - 1)}  \) (since \( \lambda < 1 \) we have that \( e^{x(\lambda - 1)}  \) is bounded and continuous, and hence Reimann integrable on [0, t] for \( t \in \mathbb{R} \) with \( t > 0 \)). Using calculus,
\begin{align*}
	\int_{0}^{t} e^{x(\lambda - 1)}   &= \left [ \frac{e^{(\lambda - 1)x} }{\lambda - 1} \right ]_{0}^{t} = \frac{e^{t(\lambda - 1)} - e^{0}  }{\lambda - 1}  = \frac{e^{t(\lambda - 1)} - 1 }{\lambda - 1}, \tag{this is the Reimann integral}
\end{align*}
so that \[\int_{(0,\infty)}e^{x(\lambda - 1)}  = \lim_{{t} \to {\infty}} \int_{[0,t]} e^{x(\lambda - 1)} = \lim_{{t} \to {\infty}} \frac{e^{t(\lambda - 1)} - 1}{\lambda - 1} = \lim_{{t} \to {\infty}} \frac{\frac{1}{e^{t(1-\lambda)} } - 1 }{\lambda - 1} = -\frac{1}{\lambda - 1} < \infty,   \] since \( 1 - \lambda > 0 \) so that \( e^{t(1-\lambda)} \to \infty \) as \( t \to \infty \); hence \( t/e^{t(1 - \lambda)} \to 0 \) as \( t\to \infty \).

Therefore, \( e^{x(\lambda - 1)}  \) is integrable. Furthermore, since for \( x \in (0,\infty) \) we have \[ \lim_{{k} \to {\infty}} ((1-(x/k))^{k}e^{\lambda x}) = e^{\lambda x} \lim_{{k} \to {\infty}} (1-(x/k))^{k} = e^{\lambda x }\cdot e^{-x}     .   \] Thus, by the dominated convergence theorem, we have that \[\lim_{{k} \to {\infty}} \int_{(0,k)} \left ( {1 - \frac{x}{k} } \right )^{k} e^{\lambda x} = \int_{(0,\infty)} \left (\lim_{{k} \to {\infty}}\left (1-\frac{x}{k} \right)^{k}e^{\lambda x}  \right)  = \int_{(0,\infty)} e^{x(\lambda - 1)} = -\frac{1}{\lambda - 1}.    \]
Therefore, we conclude that for each \( \lambda \in (-\infty,1) \), \[\boxed{\lim_{{k} \to {\infty}} \int_{(0,k)} \left ( {1 - \frac{x}{k} } \right )^{k} e^{\lambda x} = \frac{1}{1 - \lambda}   }\] \qed \\

\noindent \textbf{Lemma 6.1.} \emph{The sequence defined by \( x_{k} \coloneqq (1 - \frac{x}{k})^{k}  \) is increasing and converges to \( e^{-x}  \) for \( x \in \mathbb{R} \).}  \\
\textbf{Proof.} The convergence of \( x_{k}  \) is trivially proven using techniques from calculus. Let \( f(k) = (1- \frac{x}{k} )^{k}  \) for each \( k \in \mathbb{R} \). Then for each \( k \in \mathbb{N}  \), \( x_{k} = f(k) \) so it remains to be shown that \( \lim_{{k} \to {\infty}} f(k) = e^{-x}  \). If we let \( L = \lim_{{k} \to {\infty}} f(k)  \), by the continuity of the natural logarithm, we obtain
\begin{align*}
	\ln L &= \ln \lim_{{k} \to {\infty}} \left ( {1 - \frac{x}{k} } \right )^{k} = \lim_{{k} \to {\infty}} \ln \left (\left ( {1 - \frac{x}{k} } \right )^{k}\right ) = \lim_{{k} \to {\infty}} \frac{\ln \left ( {1 - \frac{x}{k} } \right ) }{\frac{1}{k} } \\
	      &=  \lim_{{k} \to {\infty}} \frac{\frac{\frac{x}{k^{2} } }{1-\frac{x}{k} } }{\frac{-1}{k^{2} } }   = \lim_{{k} \to {\infty}} \frac{-x}{1 - \frac{x}{k} } = -x ,  \tag{Indt. $\frac{0}{0}$ (H)}
\end{align*}
so that \( \ln L = -x \implies e^{\ln L} = L = e^{-x}   \) as needed. It remains to be shown that \( x_{k}  \) is increasing, that is, that for each \( k \in \mathbb{N} \), \( x_{k+1} / x_{k} \geq 1  \). Let \( k \in \mathbb{N}  \) be fixed. Then,
\begin{align*}
	\frac{x_{k+1} }{x_{k} }  &= \frac{\left ( {1 - \frac{x}{k+1} } \right ) ^{k+1} }{\left ( {1 - \frac{x}{k} } \right ) ^{k} } = \left ( {\frac{1 - \frac{x}{k+1} }{1 - \frac{x}{k} } } \right ) ^{k+1} \left ( {1 - \frac{x}{k} } \right ) = \left ( {\frac{\frac{k+1-x}{k+1} }{\frac{k-x}{k} } } \right )^{k+1} \left ( {1 - \frac{x}{k} } \right ) \\
				 &= \left ( {\frac{k(k+1 - x)}{(k+1) (k - x)}} \right )^{k} \left ( {1 - \frac{x}{k} } \right )  = \left ( {\frac{k^2 + -kx +k - x + x}{(k+1)(k - x)  }} \right )^{k+1} \left ( {1 - \frac{x}{k} } \right )   \\
				 &= \left ( {\frac{(k+1)(k - x) +  x}{(k+1)(k-x) }} \right )^{k+1} \left ( {1 - \frac{x}{k} } \right )  = \left ( {1 + \frac{x}{k - x}\cdot\frac{1}{k+1} } \right )^{k+1} \left ( {1 - \frac{x}{k} } \right ) \\
				 &\geq \left ( {1 + \frac{x(k+1)}{(k-x)(k+1)} } \right ) \left ( {1- \frac{x}{k} } \right ) = \left ( {1 + \frac{x}{k-x} } \right )\left ( {1 - \frac{x}{k} } \right ) \tag {Bernoulli inequality with $r = k+1, \ x = \frac{x}{(k-x)(k+1)}$} \\
				 &= 1 + \frac{x}{k-x} - \frac{x}{k} - \frac{x^{2} }{k(k-x)} = 1 + \frac{kx - x(k-x) - x^{2} }{k(k-x)} = 1. 
\end{align*}
Thus \( x_{k}  \) is increasing and converges to \( e^{-x}  \) as needed. \qed
