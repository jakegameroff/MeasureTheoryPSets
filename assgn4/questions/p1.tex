%! TeX root: ../main.tex
\noindent \textbf{Problem 1.} Let \( A \) be a measurable subset of \( \mathbb{R}^{d}  \) and \( f : A \to \overline{\mathbb{R}} \). For every \( \delta = (\delta_1, \hdots , \delta_d ) \in (0, \infty)^{d} \) and \( y = (y_1, \hdots , y_d) \in \mathbb{R}^{d}  \), let \[A_{\delta , y} \coloneqq \{ (\delta_1 x_1 + y_1, \hdots , \delta_d x_d + y_d ) : x = (x_1, \hdots , x_d) \in A \}   \] and \( f_{\delta ,y} : A_{\delta ,y} \to \overline{\mathbb{R}} \) be the function defined by \[f_{\delta ,y} (x) = f\left ( {\left ((x_1-y_1)/\delta_1 , \hdots , (x_d - y_d) / \delta_d  \right)),} \right \ \forall x \in A_{\delta,y}.  \] Show that:
\begin{enumerate}
		\item \( f \) is measurable if and only if \( f_{\delta ,y}  \) is measurable.
		\item \( f \) is integrable over \( A \) if and only if \( f_{\delta ,y}  \) is integrable over \( A_{\delta ,y}  \). Furthermore, if \( f \) is integrable over \( A \), then \[\int_{A_{\delta ,y} } f_{\delta ,y} = \delta_1\cdots \delta_d \int_{A} f.\] 	
\end{enumerate}
\textbf{Proof of 1.1.} We note that throughout the solution, by \( 1/\delta  \) we mean \( (1/\delta_1, \hdots , 1/\delta_d) \) and by \( \delta x + y \) we mean \( (\delta_1x_1+y_1 , \hdots , \delta_{d} x_{d} + y_{d}  )\). For ``\(\Rightarrow\)", suppose \( f \) is measurable. Let \( c \in \mathbb{R} \) be given. Let \( \delta x + y \in A_{\delta ,y}  \) be such that \( \delta x + y \in f_{\delta ,y} ^{-1} ([-\infty,c))  \). Then
\begin{align*}
\delta x + y \in f_{\delta ,y} ^{-1} ([-\infty,c))   &\iff f\left ( {\left ( {\frac{\delta_1x_1+y_1-y_1}{\delta_1} , \hdots , \frac{\delta_dx_d+y_d-y_d}{\delta_d} } \right ) } \right ) <c \\
						     &\iff f(x_1, \hdots , x_d) < c \iff x \in f ^{-1} ([-\infty,c)) \\
						     &\iff \delta x + y \in \delta(f ^{-1} ([-\infty,c)) ) + y \coloneqq \{ \delta x + y : x \in f ^{-1} ([-\infty, c))  \}.
\end{align*}
Thus, it follows that \( f_{\delta ,y} ^{-1} ([-\infty,c)) =  \delta(f ^{-1} ([-\infty,c)) ) + y  \). Since \( f \) is measurable, \( f ^{-1} ([-\infty,c))  \) is measurable by definition. By question three of assigment one, it follows that \(   \delta(f ^{-1} ([-\infty,c)) ) + y  = f_{\delta ,y} ^{-1} ([-\infty,c)) \) is measurable as we showed that translations and dilations of measurable sets preserve measurability. Since \( c \) was arbitrary, we conclude that \( f_{\delta ,y} ^{-1} ([-\infty,c))  \) is measurable. \\

\noindent On the other hand, for ``$\Leftarrow$", suppose \( f_{\delta ,y}  \) is measurable and let \( c \in \mathbb{R} \) be arbitrary. Then
\begin{align*}
x= (x_1, \hdots , x_d) \in f ^{-1} ([-\infty,c))  &\iff {\left ( {\frac{\delta_1 x_1 + y_1 - y_1}{\delta_1}, \hdots , \frac{\delta_d x_d + y_d - y_d}{\delta_d} } \right ) } \in f ^{-1} ([-\infty,c)) \\
					       &\iff f\left ( {\left ( \frac{(\delta_1 x_1 + y_1) - y_1}{\delta_1}, \hdots , \frac{(\delta_d x_d + y_d) - y_d}{\delta_d}   \right)} \right ) < c \\
					       &\iff \delta x + y \in f_{\delta ,y} ^{-1} ([-\infty,c)) \iff \delta x \in f_{\delta ,y} ^{-1} ([-\infty,c))  - y \\
					       &\iff x \in \frac{1}{\delta } (f_{\delta ,y} ^{-1} ([-\infty,c)) - y),
\end{align*}
where \( \frac{1}{\delta } (f_{\delta ,y} ^{-1} ([-\infty,c)) - y) \coloneqq \{ \frac{1}{\delta } (x-y) : x \in f_{\delta ,y} ^{-1} ([-\infty,c))  \}  \). Thus since \( 1/\delta \in (0,\infty)^{d}  \) and \( -y \in \mathbb{R}^{d}  \), the measurability of \( f_{\delta ,y} ^{-1} ([-\infty,c))  \) implies that \( \frac{1}{\delta } (f_{\delta ,y} ^{-1} ([-\infty,c)) - y) = f ^{-1} ([-\infty,c))   \) is measurable (this likewise follows via question one of assigment 3). Since \( c \) was arbitrary, \( f \) is measurable. \\

\noindent Therefore, we conclude that \( f \) is measurable if and only if \( f_{\delta ,y}  \) is measurable. \qed \\

\noindent \textbf{Proof of 1.2.} For \( ``$\Rightarrow$" \), suppose \( f  \) is integrable on \( A \). Let \( E \subseteq A \) be measurable. We first consider the case where \( f \) is a characteristic function, so we define \( f \coloneqq \chi_{E}  \). It follows that \( f_{\delta ,y} = \chi_{E_{\delta ,y} }  \) since for \( \delta x + y \in E_{\delta ,y}  \) we have \( f_{\delta ,y } (\delta x + y) = f(\frac{(\delta x + y) - y }{\delta } ) = f(x) = 1 \) since \( \delta x + y \in E_{\delta ,y} \implies x \in E \); on the other hand, if \( \delta x + y \notin E_{\delta ,y}  \), then \( x \notin E \) so that \(  f_{\delta ,y } (\delta x + y) =f (x) = 0  \). Furthermore, \( E_{\delta ,y}  \) is measurable and \( m(E_{\delta ,y} ) = \delta_1\cdots\delta_{d}m(E )  \) (cf. Assigment 1, Question 3). Thus, by definition of the Lebesgue integral we have \[\int_{E_{\delta ,y} } f_{\delta ,y} = \int_{E_{\delta ,y} } \chi_{E_{\delta ,y} } = m(E_{\delta ,y} ) = \delta_1\cdots\delta_{d}m(E) = \delta_1\cdots\delta_{d} \int_{E} \chi_{E} = \delta_1\cdots\delta_{d} \int_{E} f.\] Now suppose \( f \) is any characteristic function with canonical form \( f = \sum_{k=1}^{N}c_{k} \chi_{E_{k} }  \). Thus, \( f_{\delta ,y} = \sum_{k=1}^{N}c_{k} \chi_{E_{k, \delta ,y} }  \), where \( E_{k,\delta ,y} \coloneqq \{ \delta x + y : x \in E_{k}  \}  \) and \( m(E_{k,\delta ,y} ) = \delta_1\cdots\delta_{d}m(E_{k} ) \) (cf. Assignment 1, Question 3). Therefore, by the construction of the Lebesgue integral of simple functions, we have \[\int_{E_{\delta ,y} } f_{\delta ,y} = \int_{E_{\delta ,y} } \sum_{k=1}^{N}c_{k}\chi_{E_{k,\delta ,y} } = \sum_{k=1}^{N}c_{k} m(E_{k,\delta ,y} ) = \delta_1\cdots\delta_{d} \sum_{k=1}^{N}c_{k} m(E_{k}) = \delta_1\cdots\delta_{d} \int_{E} \sum_{k=1}^{N}c_{k}\chi_{E} = \delta_{1}\cdots\delta_{d}  \int_{E} f.\] Thus, we have the desired equality for simple functions.

We now suppose that \( f \) is non-negative on \( E \). In this case, so is \( f_{\delta ,y}  \) since for each \( \delta x + y \in E_{\delta ,y}  \), \( f_{\delta ,y} (\delta x + y ) = f(x) \geq 0 \). By the simple approximation theorem, there exists a sequence \( \{ \varphi _{k}  \}_{k \in \mathbb{N} }   \) of simple functions such that \( \lim_{{k} \to {\infty}} \varphi _{k} = f \) and for each \( k \in \mathbb{N}  \), \( \varphi _{k} \geq 0\) and \( \varphi _{k} \leq \varphi _{k+1}  \). Define for each \( k \in \mathbb{N}  \) and \( \delta x + y \in E_{\delta ,y}  \) the function \( \varphi_{k,\delta ,y} (\delta x + y) = \varphi _{k} (\frac{(\delta x + y )-y}{\delta } ) = \varphi _{k} (x) \) so that:
\begin{enumerate}
	\item For each \( \delta x + y \in E_{\delta ,y}  \), \( \varphi _{k,\delta ,y} (\delta  x + y) \to f_{\delta ,y} (\delta x + y) \). Indeed, \( \varphi _{k,\delta ,y} (\delta x + y) = \varphi _{k} (x) \to f(x) = f_{\delta ,y} (\delta x + y ) \).
\item For each \( k \in \mathbb{N}  \), \( 0 \leq \varphi _{k,\delta ,y} \leq \varphi _{k+1,\delta ,y}  \). Indeed, for each \( k \in \mathbb{N}  \) and \( \delta x + y \in E_{\delta ,y}  \) we have \[ 0 \leq \varphi _{k}(x)= \varphi _{k,\delta ,y} (\delta x + y) \leq \varphi _{k+1}(x) = \varphi _{k+1,\delta ,y} (\delta x + y).     \]
\end{enumerate}
Thus, we can use the monotone convergence theorem as well as the previous argument regarding simple functions to obtain \[\int_{E_{\delta ,y} } f_{\delta ,y} = \lim_{{k} \to {\infty}} \int_{E_{\delta ,y} } \varphi _{k,\delta ,y}  = \lim_{{k} \to {\infty}}\left (\delta_{1}\cdots\delta_{d}   \int_{E} \varphi _{k}  \right) = \delta_1\cdots\delta_{d} \lim_{{k} \to {\infty}} \int_{E} \varphi _{k} = \delta_1\cdots\delta_{d} \int_{E}f.\]

Finally, if \( f \) can change sign, we write \( f = f_{+} - f_{-}  \). It follows that for \( \delta x + y \in E_{\delta ,y}  \) we have that \( f_{\delta ,y}(\delta x + y) = f(\frac{(\delta x + y) - y}{\delta }) = f_{+}(\frac{(\delta x + y) - y}{\delta }) - f_{-}(\frac{(\delta x + y) - y}{\delta }) =: f_{+,\delta ,y} (\delta x + y )  - f_{-,\delta ,y} (\delta x + y)    \), where $f_{+,\delta ,y}(\delta x + y) \coloneqq f_{+}(\frac{(\delta x + y) - y}{\delta }) $ and \( f_{-,\delta ,y} (\delta x + y) \coloneqq f_{-}(\frac{(\delta x + y) - y}{\delta }) \). Thus, by the previous argument regarding non-negative functions, since \( f_{+},f_{+,\delta ,y},f_{-},f_{-,\delta ,y}     \geq 0 \) on their domains. Since \( \int_{E} f = \int_{E} f_{+} - \int_{E} f_{-}  \), it follows that \[\int_{E_{\delta ,y} } f_{\delta ,y} = \int_{E_{\delta ,y} } f_{+,\delta ,y} - \int_{E_{\delta ,y} } f_{-,\delta ,y} = \delta_1\cdots\delta_{d} \int_{E } f_{+} - \delta_1\cdots\delta_{d} \int_{E}f_{-} = \delta_1\cdots\delta_{d} \int_{E} f.       \] Thus, in all of such cases, we have that \( f_{\delta ,y}  \) is also integrable since \( \int_{E_{\delta ,y} } f_{\delta ,y} = \delta_1\cdots\delta_{d} \int_{E} f < \infty  \), thereby completing the forward implication of the proof, taking \( E = A \).\\

\noindent On the other hand, for ``$\Leftarrow$", suppose \( f_{\delta ,y}  \) is integrable over \( A_{\delta ,y}  \). For the sake of clarity, let \( E \coloneqq A_{\delta ,y}  \), and we denote \( f_{\delta ,y}  \) by \( g \). By the forward implication, we know that for any \( \alpha = (\alpha_1, \hdots , \alpha_{d} ) \in (0,\infty)^{d}  \) and \( z = (z_1, \hdots , z_{d}) \in \mathbb{R}^{d}  \), the function \( g \circ h \) is integrable, where \( h : E_{\alpha,z} \to \overline{\mathbb{R}}  \) is a function given by \( h(x) = \frac{x-z}{\alpha} \). Thus, for the given \( \delta  \) and \( y \), let \( \alpha \coloneqq 1/\delta  \) and \( z \coloneqq -y/\delta  \) so that by \( ``$\Rightarrow$" \) we obtain that \( g \circ h \) is integrable. But notice that \(A = E_{\alpha, z} \) as \[ \alpha(\delta x + y) + z \in E_{\alpha,z} \iff  \frac{1}{\delta } (\delta x + y) - \frac{y}{\delta } = x + \frac{y}{\delta } - \frac{y}{\delta } = x \in E_{\alpha,z}     \iff \delta x + y \in E = A_{\delta ,y}   \iff x \in A. \] Thus, \( h : A \to \overline{\mathbb{R}} \) and \( g \circ h = f_{\delta ,y} \circ h \) is integrable. But for \( x \in A \), \[ f_{\delta , y } \circ h (x) = f_{\delta ,y} (h(x)) = f_{\delta ,y} \left(\frac{x-z}{\alpha} \right)  = f_{\delta ,y} \left ( \frac{x + \frac{y}{\delta } }{\frac{1}{\delta } }  \right ) = f\left ( {\frac{\left ( {\frac{x+\frac{y}{\delta } }{\frac{1}{\delta } } } \right ) - y }{\delta } } \right ) = f\left ( {\frac{\delta x + y - y}{\delta } } \right )  = f(x). \] Thus, we conclude that \( g\circ h = f \) is integrable, thereby completing the proof. \qed
