%! TeX root: ../main.tex
\noindent \textbf{Problem 2.} Let \( f \) be integrable over \( A \subseteq \mathbb{R}^{d}  \). Show that:
\begin{enumerate}
	\item If \( \int_{B} f \geq 0 \) for all measurable sets \( B \subseteq A \), then \( f \geq 0  \) a.e. in \( A \).
	\item If \( \int_{B} f = 0  \) for all measurable sets \( B \subseteq A \), then \( f = 0 \) a.e. in \( A \).
\end{enumerate}
\noindent \textbf{Proof of 2.1.} Define \( N = \{ x \in A : f(x) < 0 \} = f ^{-1} ([-\infty,0))  \subseteq A\). Since \( f \) is integrable, it is measurable; thus \( N \) is measurable by the measurability of \( f \). Since \( N \subseteq A \), by hypothesis \[\int_{N} f \geq 0.\] Notice that for each \( k \in \mathbb{N}  \), \( k\cdot\chi_{N} f \leq f \) on \( A \) since for \( x \in N \), the \( k\cdot \chi_{N}(x) f(x) = -k |f(x)| \leq - |f(x)|\), and for \( x \notin N \), \( k\cdot \chi_{N} (x) f(x) = 0 \leq f(x) \), since \( f \geq 0 \) on \( A \setminus N \). Thus, by monotonicity, for every \( k \in \mathbb{N}  \) we have
\begin{align*}
	\int_{A} k \cdot \chi_{N} f \leq \int_{A} f &\iff k \int_{A} \chi_{N} f \leq \int_{A} f \tag{by linearity} \\
						    &\iff \int_{N} f = \int_{A} \chi_{N} f \leq \frac{1}{k} \int_{A} f.
\end{align*}
Thus, we have that for every \( k \in \mathbb{N} \), \( 0 \leq \int_{N} f \leq \frac{1}{k} \int_{A} f \). Since \(f \) is integrable, there exists \( \ell \in \mathbb{R} \) such that \( \int_{A} f = \ell. \) Thus, \( 0 \leq \int_{N} f \leq \ell / k \). Since \( \ell \) is fixed, sending \( k \to \infty \) gives \( \int_{N} f \leq 0 \) so that \( \int_{N} f = 0 \), since we also have \( \int_{N} f \geq 0 \). But by definition, \( f < 0 \) on \( N \) so that \( f_{+} = \max(f,0) = 0 \) on \( N \). Hence, \[0 = \int_{N} f = \int_{N} f_{+} - \int_{N} f_{-} = \int_{N} 0  - \int_{N} f_{-} = -\int_{N} f_{-}  = - \left ( {\underbrace{\int_{A \setminus N} f_{-}}_{=0}   + \int_{N} f_{-} } \right ) = -\int_{A} f_{-},\] which holds as \( f_{-} = 0 \) on \( A \setminus N \) (as \( f \geq 0 \) on \( A \setminus N \)) and since \( (A \setminus N) \cap N = \emptyset  \) and the sets \( A \setminus N \), \( N \) are measurable (difference of measurable sets; for \( B_1,B_2\subseteq \mathbb{R}^{d}  \) disjoint, measurable, \( f \) integrable: \( \int_{B_1\cup B_2} f = \int_{B_1} f+ \int_{B_2} f \)). Thus, we conclude that \( -\int_{A} f_{-} = 0 \) so that \( \int_{A} f_{-} = 0 \). Since \( f_{-} : A \to [0,\infty] \) is measurable and \( \int_{A} f_{-} = 0 \), by lecture we have \( f_{-} = 0 \) a.e. in \( A \). Thus, \( \max(-f, 0) = 0 \) a.e. in \( A \) so that \( f \geq 0 \) a.e. in \( A \) as desired. \qed

\noindent \textbf{Proof of 2.2.} By (2.1), we already know that \( f \geq 0 \) a.e. in \( A \). Thus, it remains to be shown that \( f \leq 0 \) a.e. in \( A \). Define \( N = \{ x \in A : f(x) > 0 \} = f ^{-1} ((0,\infty]) \subseteq A \), which is measurable since \( f \) is. Thus, by hypothesis, \( \int_{N} f = 0 \) so that, since \( f_{-} = 0 \) on \( N \) (since \( f \) is positive on \( N \)) and \( f_{+} = 0 \) on \( A\setminus N \) (as \( f \) non-positive on \( A \setminus N \)), \[\int_{N} f = \int_{N} f_{+} - \int_{N} f_{-} = \int_{N} f_{+} = \left ( {\underbrace{\int_{A \setminus N} f_{+}}_{=0}  + \int_{N} f_{+} } \right ) = \int_{A} f_{+} = 0,\] since \( (A \setminus N) \cap N = \emptyset  \) and the sets \( A \setminus N \), \( N \) are measurable (difference of measurable sets; property of integral over disjoint, measurable sets as above). Thus, since \( f_{+} : A \to [0,\infty]  \) is measurable and \( \int_{A} f_{+} = 0  \), it follows by lecture that \( f_{+} = 0 \) a.e. in \( A \) so that \( f \leq 0 \) a.e. in \( A \).

Since \( f \geq 0 \) a.e. in \( A \), the set \( N_1 \coloneqq \{ x \in A : f(x) < 0 \}  \) has measure zero, and since \( f \leq 0 \) a.e. in \( A \), the set \( N_{2} \coloneqq \{ x \in A : f(x) > 0 \}   \) has measure zero. Thus, the set \( M = N_1 \cup N_2\) has measure zero since by finite sub-additivity, \[m(M) = m(N_1 \cup N_2) \leq m(N_1) + m(N_2) = 0 \implies m(M) = 0.\] Thus, the set on which \( f \neq 0 \) has measure zero. Hence, we conclude that \( f = 0 \) a.e. in \( A \). \qed
