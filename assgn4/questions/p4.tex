%! TeX root: ../main.tex
\noindent \textbf{Problem 4.} Let \( f \) be measurable on \( A \subseteq \mathbb{R}^{d}  \) and \( (A_{k} )_{k \in \mathbb{N} }  \) be a sequence of measurable subsets of \( A \). Show that:
\begin{enumerate}
	\item If \( f \) is non-negative and \( A_{k} \subseteq A_{k+1}  \) for all \( k \in \mathbb{N}  \) then \[\int_{\bigcup_{k=1}^{\infty} A_{k} } f = \lim_{{k} \to {\infty}} \int_{A_{k} } f.\]
	\item If \( f \) is integrable and \( A_{k} \subseteq A_{k+1}  \) for all \( k \in \mathbb{N}  \) then \[\int_{\bigcup_{k=1}^{\infty} A_{k} } f = \lim_{{k} \to {\infty}} \int_{A_{k} } f.\]
	\item If \( f \) is integrable over \( A_1 \) and \( A_{k+1} \subseteq A_{k}  \) for all \( k \in \mathbb{N}  \), then \[\int_{\bigcap_{k=1}^{\infty} A_{k} } f = \lim_{{k} \to {\infty}} \int_{A_{k} } f.\] 
\end{enumerate}
\noindent \textbf{Proof of 4.1.} Let \( f \) be non-negative. We note that if \( f \) is integrable over \( \bigcup_{k=1}^{\infty} A_{k}  \), then we obtain from (2) that \[\int_{\bigcup_{k=1}^{\infty} A_{k} } f = \lim_{{k} \to {\infty}} \int_{A_{k} } f\] as needed, thus we suppose that \( \int_{\bigcup_{k=1}^{\infty} A_{k} } f = \infty \), as this is all that is left to show by non-negativity. To show that \( \lim_{{k} \to {\infty}} \int_{A_{k} } f = \int_{\bigcup_{k=1}^{\infty} A_{k} } = \infty \), we show that it is unbounded. Thus, let \( M \in \mathbb{N}  \) be arbitrary. Since \( A_{k} \subseteq A_{k+1}  \) for each \( k \in \mathbb{N}  \), it follows that there is a large enough \( n \in \mathbb{N}  \) such that \[ \int_{\bigcup_{k=1}^{n} A_{k} } f > M; \] indeed, if there is no such \( n \), then since limits respect order \[\lim_{{n} \to {\infty}} \int_{\bigcup_{k=1}^{n} A_{k} } f = \int_{\bigcup_{k=1}^{\infty} A_{k} } f \leq M < \infty\] is a contradiction to the choice of \( f \). We note that \( \bigcup_{k=1}^{n} A_{k} = A_{n}  \) (since \( A_{n} \subseteq \bigcup_{k=1}^{n} A_{n}  \) trivially and \( x \in \bigcup_{k=1}^{n} A_{k}   \implies \exists m = 1, 2, \hdots , n : x \in A_{m} \subseteq A_{m+1}\subseteq \cdots \subseteq A_{n} \implies x \in A_{n}   \) as needed.) Thus, for this given \( M \), we have found a set \( A_{n}  \) such that \[\int_{\bigcup_{k=1}^{n} A_{k} }f = \int_{A_{n} } f > M .\] Since \( M \) was arbitrary, we conclude as required that \[\int_{\bigcup_{k=1}^{\infty} A_{k} } f = \lim_{{k} \to {\infty}} \int_{A_{k} } f = \infty.\] \qed

\newpage\noindent \textbf{Proof of 4.2.} Let \( f \) be integrable over \( A = \bigcup_{k=1}^{\infty} A_{k}  \) and define \( E_1 \coloneqq A_1 \) and for \( k \in \mathbb{N}_{\geq 2}  \), let \( E_{k} \coloneqq A_{k} \setminus A_{k-1}  \). Then each \( E_{k}  \) is a difference of measurable sets and hence measurable, and \( \bigcup_{k=1}^{\infty} A_{k} = \bigcup_{k=1}^{\infty} E_{k}  \).\footnote{``$\supseteq$" clearly holds by construction. On the other hand, if \( x \in A_{k}  \) for some \( k \in \mathbb{N}_{\geq 2}  \) then \( x \in A_{k} \setminus A_{k-1}  \) or \( x \in A_{k-1}  \); in the former case, \( x \in E_{k} \subseteq \bigcup_{k=1}^{\infty} E_{k}   \) and in the latter case, \( x \in A_{k-1} \setminus A_{k-2}  \) or \( x \in A_{k-2}  \); eventually, \( x \in E_{j} \subseteq \bigcup_{k=1}^{\infty} E_{k} \) for some \( j \geq 2 \) (otherwise \( x \in A_1=E_1 \)). And if \( x \in A_1 \) then \( x \in E_1 \subseteq \bigcup_{k=1}^{\infty} E_{k}  \). This set equality also holds by the exact same reasoning with \( \bigcup_{k=1}^{n} A_{k} = \bigcup_{k=1}^{n} E_{k}  \) for each \( n \in \mathbb{N}  \).} By construction, the \( E_{k}  \)'s are mutually disjoint. Thus, it follows by disjointness that for each \( n \in \mathbb{N}  \)
\begin{align*}
	\int_{\bigcup_{k=1}^{n} A_{k}  } f = \int_{\bigcup_{k=1}^{n} E_{k} } f = \sum_{k=1}^{n} \int_{E_{k} } f = \int_{A_{n} } f  \tag{Lemma 4.1 by disjointness of \( E_{k} \)'s.}
\end{align*}
since \( A_{k} \subseteq A_{k+1} \ \forall k \in \mathbb{N} \implies A_{n} = \bigcup_{k=1}^{n} A_{k}  \) for \( n \in \mathbb{N}  \). Thus, again by disjointness and Lemma 4.2, we have
\begin{align*}
	\int_{\bigcup_{k=1}^{\infty} A_{k} } f &= \int_{\bigcup_{k=1}^{\infty} E_{k} } f= \lim_{{n} \to {\infty}} \int_{\bigcup_{k=1}^{n}E_{k}  }  f = \lim_{{n} \to {\infty}} \sum_{k=1}^{n}\int_{E_{k} } f = \lim_{{n} \to {\infty}} \int_{A_{n} } f.  
\end{align*}
Replacing \( n \) with \( k \) gives \( \int_{\bigcup_{k=1}^{\infty} A_{k} } f = \lim_{{k} \to {\infty}} \int_{A_{k} } f \), completing the proof. \qed \\

\noindent \textbf{Proof of 4.3.} 

Let \( f  \) be integrable over \( A_1 \) so that \( \int_{A_1} f < \infty \). Define for all integers \( k \geq 1 \) \( E_{k} \coloneqq A_1 \setminus A_{k}  \). Then since \( A_{k+1} \subseteq A_{k}  \) for each \( k \in \mathbb{N}  \), we must have that the \( E_{k}  \)'s are mutually disjoint.\footnote{Indeed, if \( 1 \leq i < j \) are integers, then \( E_{i} = A_1 \setminus A_{i}  \) and \( E_{j} = A_1 \setminus A_{j}  \); hence \( i < j \implies A_{j} \subseteq A_{i}  \) so that \( E_{i} \cap E_{j} = \emptyset  \).} Moreover, since for each \( k \in \mathbb{N}  \) we have \( A_{k+1} \subseteq A_{k}  \), we obtain \( E_{k} = A_1 \setminus A_{k}  \subseteq A_1 \setminus A_{k+1} =  E_{k+1}  \) for every \( k \in \mathbb{N}  \). Furthermore, \[\bigcup_{k=1}^{\infty} E_{k} = \bigcup_{k=1}^{\infty} (A_1 \setminus A_{k} ) = \bigcup_{k=1}^{\infty} (A_1 \cap A_{k}^{c} ) = A_1 \cap \bigcup_{k=1}^{\infty} A_{k}^{c} = A_1 \setminus \left ( {\bigcup_{k=1}^{\infty} A_{k} ^{c} } \right )^{c} = A_1 \setminus \bigcap_{k=1}^{\infty} A_{k}    \] via DeMorgan's law. Since \( f \) is integrable on \( A_1 \), it is also integrable on \( \bigcup_{k=1}^{\infty} E_{k} = A_1 \setminus \bigcap_{k=1}^{\infty} A_{k} \subseteq A_1 \). By (2), since the \( E_{k}  \)'s increase towards their union, we obtain \[\int_{A_1 \setminus \bigcap_{k=1}^{\infty} A_{k} } f =  \int_{\bigcup_{k=1}^{\infty} E_{k}} f = \lim_{{k} \to {\infty}} \int_{E_{k} } f = \lim_{{k} \to {\infty}} \int_{A_1 \setminus A_{k} } f. \tag{$\ast$}  \] Furthermore, since \( f \) is integrable on \(A_1 \), it is integrable on \( \bigcap_{k=1}^{\infty} A_{k} \subseteq A_1 \) so that \[ \int_{A_1 \setminus \bigcap_{k=1}^{\infty} A_{k} } f = \int_{A_1} f - \int_{\bigcap_{k=1}^{\infty} A_{k} } f ,\] and since \( A_{k} \subseteq A_1 \) for each \( k \in \mathbb{N}  \), we also have \( \int_{A_1 \setminus A_{k} } f = \int_{A_1} f - \int_{A_{k} } f  \) (cf. Lemma 4.3). Therefore, using \((\ast)\) we obtain that \[\int_{A_1} f - \int_{\bigcap_{k=1}^{\infty} A_{k} } } f = \int_{A_1 \setminus \bigcap_{k=1}^{\infty} A_{k} } =  \lim_{{k} \to {\infty}} \left ( {\int_{A_1 \setminus A_{k} } } f \right )  = \lim_{{k} \to {\infty}} \left ( {\int_{A_1} f - \int_{A_{k} } f} \right ) = \int_{A_1} f - \lim_{{k} \to {\infty}} \int_{A_{k} } f. \] Since \( \int_{A_1} f < \infty \), we obtain (by subtracting and multiplying by \( -1 \)) \[\int_{\bigcap_{k=1}^{\infty} A_{k} } f = \lim_{{k} \to {\infty}} \int_{A_{k} } f,\] thereby completing the proof.  \qed









\noindent \textbf{Lemma 4.1.} Let \( X \subseteq \mathbb{R}^{d}  \) and \( f \) be an integrable function on \( X \). If \( \{ X_{k}  \}_{k = 1}^{n}   \) is a finite collection of mutually disjoint measurable subsets of \( X \), then \[\int_{X_1 \cup \cdots \cup  X_{n} } f = \int_{X_1} f + \cdots + \int_{X_{n} } f. \]

\noindent \emph{Proof.} We proceed by induction on \( n \). For \( n = 1 \) there is nothing to prove. If \( n = 2 \), then \( \int_{X_1 \cup X_2} f = \int_{X_1} f + \int_{X_{2} } f \) was proven in lecture. Suppose the claim holds for some \( n \geq 1 \) and let \( X_1, \hdots , X_{n}  , X_{n+1} \) be a collection of mutually disjoint measurable subsets of \( X \). Then letting \( Y_{k} \coloneqq X_{k}   \) for \( k = 1, 2, \hdots , n - 1 \) and \( Y_{n} \coloneqq X_{n} \cup X_{n+1}    \), the inductive hypothesis gives:
\begin{align*}
	\int_{Y_1 \cup \cdots \cup Y_{n} } f  &= \int_{Y_1} f + \cdots + \int_{Y_{n} } f = \int_{X_1} f + \cdots + \int_{X_{n-1} } f  + \int_{X_{n } \cup X_{n+1}} f \\
					      &= \sum_{k=1}^{n-1} \int_{X_{k} } f + \int_{X_{n} \cup X_{n+1} } f = \sum_{k=1}^{n+1}\int_{X_{k} } f \tag{by disjointness and case \( n = 2 \)}.
\end{align*}
Furthermore, notice that \[\int_{X_1 \cup \cdots X_{n+1} } f = \int_{Y_1 \cup \cdots \cup Y_{n} }f = \sum_{k=1}^{n+1}\int_{X_{k} } f, \] which completes the inductive step and hence the proof.

\noindent \textbf{Lemma 4.2.} Let \( X \subseteq \mathbb{R}^{d}  \) and \( f \) be a integrable function on \( X \). If \( \{ X_{k}  \}_{k=1} ^{\infty}  \) is a sequence of mutually disjoint measurable subsets of \( X \) such that \[X = \bigcup_{k=1}^{\infty} X_{k}, \mbox{ then } \int_{X} f = \sum_{k=1}^{\infty} \int_{X_{k} } f . \]
\begin{proof}
	Define for \( j \in \mathbb{N}  \) \[ f_j \coloneqq f\cdot\chi_{\bigcup_{k=1}^{j} X_{k} } . \] Then for each \( j \in \mathbb{N}  \), \( f_{j}  \) is measurable by lecture as it is a product of measurable functions (note that characteristic functions are measurable by lecture). Furthermore, by the construction of \( f_{j}  \), \( |f_{j}| \leq |f|  \) since for \( x \in \bigcup_{k=1}^{j} X_{k}  \), \( |f_{j} (x)| = |f(x)|  \) and for \( x \notin \bigcup_{k=1}^{j} X_{k}  \), \( |f_{j} (x) | = 0 \leq |f(x) | \); and \( |f| \) is integrable by lecture since \( f \) is. Moreover, \( f_{j} \to f \) pointwise: given \( \varepsilon > 0 \) and \( x \in X \), since \( X = \bigcup_{k=1}^{\infty} X_{k}  \) there is a large enough \( j \) so that \( x \in \bigcup_{k=1}^{j} X_{k}  \) so that for \( j' \geq j \), \( |f_{j'}(x) - f(x)   | = |f(x)\cdot 1 - f(x)  | = 0 < \varepsilon . \) Thus, by the dominated convergence theorem we have \[\lim_{{j} \to {\infty}} \int_{X} f_{j} = \int_{X} f.\] Notice that for each \( j \in \mathbb{N}  \), since \( f_{j} = f  \) on \( \bigcup_{k=1}^{j} X_{k}  \) and using the disjointness (and measurability) of \( \bigcup_{k=1}^{j} X_{k}  \) and \( X \setminus \bigcup_{k=1}^{j}X_{k}   \), \[ \int_{X} f_{j} = \int_{\bigcup_{k=1}^{j} X_{k} } f_{j} + \underbrace{\int_{X \setminus \bigcup_{k=1}^{j} X_{k} } f_{j} }_{=0} = \int_{\bigcup_{k=1}^{j} X_{k}}f  }. \] By Lemma 4.1, since the \( X_{k}  \)'s are mutually disjoint by hypothesis, we additionally have: \[\int_{X} f_{j} = \int_{\bigcup_{k=1}^{j} X_{k} }f = \sum_{k=1}^{j} \int_{X_{k} } f.   \] Thus, we conclude that \[ \int_{X} f = \lim_{{j} \to {\infty}} \int_{X} f_{j} = \lim_{{j} \to {\infty}} \sum_{k=1}^{j}\int_{X_{k} } f_{j} = \sum_{k=1}^{\infty} \int_{X_{k} } f.  \] 
\end{proof}
\noindent \textbf{Lemma 4.3.} Let \( f \) be integrable on \( A \) and \( B \subseteq A \). Then \[\int_{A \setminus B} f = \int_{A} f - \int_{B} f.\]
\begin{proof}
Notice that \( B \subseteq A \implies  A = A \setminus B \cup B	\) is a disjoint union. Since \( f\) is integrable over \( A \), it is integrable over \( B \) so that \[\int_{A} f = \int_{A \setminus B} f + \int_{B} f \implies \int_{A \setminus B}  = \int_{A} f - \int_{B} f, \] using \( \int_{B} f < \infty \) to subtract it on both sides.
\end{proof}
