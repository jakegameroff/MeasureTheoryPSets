%! TeX root: ../main.tex
\noindent \underline{\textbf{Problem 2.}} Let \( A \subseteq \mathbb{R}^{d} 	 \) be such that \( m_*(A) < \infty \). Show that \( A \) is non-measurable if and only if there exists a \( G_\delta  \) set \( G \) such that \( A \subseteq G \), \( m(G) = m_*(A)  \), and \( m_*(G \setminus A) > 0. \) 
\begin{proof}$ $\newline
	\( [\implies ] \) Let \( A \subseteq \mathbb{R}^{d}  \) be a non-measurable set with \( m_*(A) < \infty \). We construct an appropriate \( G_\delta  \) set \( G \). 

	We use the following fact from lecture: \( m_*(A) = \inf \{ m_*(\mathcal{O}) : A \subseteq \mathcal{O}, \  \mathcal{O} \mbox{ open}  \}  \). Thus, by definition of the infimum, given \( \varepsilon > 0 \) there exists an open set \( \mathcal{O}_\varepsilon  \) containing \( A \) such that \( m_*(\mathcal{O} _\varepsilon ) < m_*(A) + \varepsilon .  \) We use this to define a sequence \( \{ \mathcal{O}_{n}  : n \in \mathbb{N}_+ \}  \) of open sets containing \( A \). Let \( \mathcal{O}_{1}  \) be an open set such that \( m_*(\mathcal{O}_{1} ) < m_*(A) + 1 \); more generally, for any \( n \geq 1 \), define \( \mathcal{O}_{n}  \) to be an open set such that \( m_*(\mathcal{O}_{n} ) < m_*(A) + \frac{1}{n}  \). We define our \( G_\delta  \) set by \[G \coloneqq \bigcap_{n=1}^{\infty} \mathcal{O}_{n}. \]
\noindent \textbf{Property 1.} \emph{\(A \subseteq G\)}. This is trivial, since by construction, for each \( n \geq 1 \), \( A \subseteq \mathcal{O}_{n}  \), we must have that \( A \subseteq \bigcap_{n=1}^{\infty} \mathcal{O}_{n} = G. \) \\

\noindent \textbf{Property 2.} \( m(G) = m_*(A)  \). Since \( A \subseteq G \), by monotonicity, \( m_*(A) \leq m_*(G) = m(G). \) It remains to be shown that \( m(G) \leq m_*(A). \) Let \( \varepsilon >0 \) be fixed. By Archimedeanity, there is an \( m \in \mathbb{N}  \) with \( 1/m < \varepsilon \) and hence a corresponding open set \( \mathcal{O}_{m}  \) from the sequence such that \( m_*(\mathcal{O}_{m} ) < m_*(A) + \frac{1}{m} < m_*(A) + \varepsilon   \). Since \( G = \bigcap_{n=1}^{\infty} \mathcal{O}_{n} \subseteq \mathcal{O}_{m} \), monotonicity imples that \( m(G) = m_*(G) \leq m_*(\mathcal{O}_{m}  ) < m_*(A) +  \varepsilon  \). Hence \( m(G) < m_*(A) + \varepsilon  \) for every \( \varepsilon > 0 \). Since \( \varepsilon  \) was arbitrary, we conclude that \( m(G) \leq m_*(A)  \) and hence \( m(G) = m_*(A)  \). \\

\noindent \textbf{Property 3.} \( m_*(G \setminus A) > 0.  \) Since \( A \) is non-measurable, there exists an \( \varepsilon > 0 \) such that for every open set \( \mathcal{O} \subseteq \mathbb{R}^{d}  \) with \( A \subseteq \mathcal{O}  \), \( m_*(\mathcal{O} \setminus A) > \varepsilon . \) For this \( \varepsilon  \), since \( m_*(G \setminus A) = \inf \{ m_*(\mathcal{O}) : G\setminus A \subseteq  \mathcal{O} , \mathcal{O} \mbox{ open}  \}   \), there is an open set \( \mathcal{O}  \) such that \( G \setminus A \subseteq \mathcal{O}  \) and \( m_*(\mathcal{O}) < m_*(G \setminus A) + \varepsilon  \). But then \( \mathcal{O} \setminus A \subseteq \mathcal{O}  \), thus by monotonicity \[m_*(\mathcal{O} \setminus A) \leq m_*(\mathcal{O} ) < m_*(G \setminus A) + \varepsilon .   \] But \( A \) being non-measurable implies \( m_*(\mathcal{O} \setminus A) > \varepsilon   \), so we obtain \[\varepsilon < m_*(\mathcal{O} \setminus A) < m_*(G \setminus A) + \varepsilon \implies m_*(G\setminus A) > 0,\] subtracting \( \varepsilon  \) on both sides. Thus, there is a \( G_\delta  \) set \( G \) such that \( A \subseteq G, \ m(G) = m_*(A), \mbox{ and } m_*(G \setminus A) > 0, \) thereby completing the forward implication. \\

\noindent [\( \impliedby \)] Suppose there is a \( G_\delta  \) set \( G \) such that \( A \subseteq G, \ m(G) = m_*(A), \mbox{ and } m_*(G \setminus A) > 0.\) We must show that \( A \) is not measurable. Hence, suppose towards contradiction that \( A \) is measurable. Thus, since \( A \subseteq G \), we can write \( G = (G \setminus A) \cup A \), a disjoint union, so that \(m(G) = m((G \setminus A) \cup A) = m(G \setminus A) + m(A),\) which holds by the finite case of countable additivity.\footnote{Note that since \( G \) and \( A  \) are measurable, \( G\setminus A = G \cap A^{c}  \) is measurable as finite intersections of measurable sets are measurable and complements of measurable sets are measurable.} Hence \( m_*(G \setminus A) = m(G \setminus A)  = m(G) - m(A) = 0 \), since \( m(G) = m_*(A) = m(A) \) by supposition; note that we can subtract \( m(A) \) on both sides since we are given that \( m_*(A)=m(A)=m(G) < \infty \). Thus, \( m_*(G \setminus A) = 0 \), a contradiction. Thus, \( A \) is not measurable. 
\end{proof}
