\include{../preamble.tex}
\begin{document}
%! TeX root: ../main.tex
\noindent \underline{\textbf{Problem 1.}} Let \( (A_k)_{k \in \mathbb{N} }  \) be a sequence of measurable sets such that \[\sum_{k=1}^{\infty}m(A_k) < \infty.\] Let \( A \) be the set of all \( x \in \mathbb{R}^{d}  \) such that \( x \in A_k \) for infinitely many \( k \). Show that \( A \) is measurable and \( m(A) = 0. \)   
\begin{proof}
It suffices to show that \( m_*(A) = 0. \) By lemma 1.1, since \( \sum_{k=1}^{\infty}m(A_k)  = \ell < \infty \) for some \( \ell \in \mathbb{R}_{\geq 0}  \), given \( \varepsilon > 0 \) there exists an \( N \in \mathbb{N}  \) such that \( \sum_{k=N}^{\infty} m(A_k) < \varepsilon . \)

Now we claim that \( A \subseteq \bigcup_{k=N}^{\infty} A_k. \) Indeed, if we fix \( x \in A \) and suppose towards contradiction that \( x \notin \bigcup_{k=N}^{\infty} A_k \), then \( x \) can only be an element of at most each \( A_k \) with \( 1 \leq k \leq N-1 \), a contradiction to the choice of \( A \) as \( x \in A_k \) for only finitely many \( k \).

Thus, by monotonicty and sub-additivity, \[m_*(A) \leq m_*\left ( {\bigcup_{k=N}^{\infty} A_k} \right ) \leq \sum_{k=N}^{\infty} m(A_k) < \varepsilon .  \] Since \( \varepsilon  \) was arbitrary, we conclude that \( m_*(A)  = 0 \). Thus, since we proved in lecture that sets of outer measure zero are measurable, we conclude that \( A \) is measurable with \( m(A) = 0. \) 





\end{proof}
\begin{proof}[Lemma 1.1] \emph{Let \( (a_n)_{n \in \mathbb{N} }  \) be a sequence of non-negative reals with \(\sum_{n=1}^{\infty}a_n = \ell < \infty\) for some \( \ell \in \mathbb{R}_{\geq 0}  \). Then for every \( \varepsilon > 0 \) there exists an \( M \in \mathbb{N}  \) such that \( \sum_{n=M}^{\infty}a_n < \varepsilon  \).}

	Let \( (a_n)_{n \in \mathbb{N} }  \) be given as above. Since \(\sum_{n=1}^{\infty}a_n = \ell \in \mathbb{R}_{\geq 0} ,\) we have that \[\lim_{{N} \to {\infty}} \sum_{n=1}^{N}a_n = \ell.\] Thus, given \( \varepsilon > 0 \) there is an \( M - 1 \in \mathbb{N}  \) such that
\begin{align*}
	\left | \ell - \sum_{n=1}^{M - 1}a_n \right | &= \ell - \sum_{n=1}^{M-1}a_n \tag{\(a_n \geq 0 \ \forall n \geq 1\) } \\ 
						      &= \ell - a_1 - a_2 - \cdots - a_{M-1} = \sum_{n=M}^{\infty}a_n < \varepsilon,
\end{align*}
as required.
\end{proof}

\newpage
%! TeX root: ../main.tex
\noindent \underline{\textbf{Problem 2.}} Let \( A \) be a subset of \( [0, \infty) \) and \( A^{2} \coloneqq \{ x^{2} \in \mathbb{R} : x \in A \}  \).
\begin{enumerate}
	\item Prove that if \( m_*(A) = 0 \), then \( m_*(A^{2} ) = 0 \).

	\item Give an example of a set \( A \) such that \( m_*(A) < \infty \) and \( m_*(A^{2} ) = \infty \).
\end{enumerate}
\begin{proof}[Proof of 2.1]
	We first suppose that \( A \) is bounded, i.e. there is an \( M \in \mathbb{N}  \) such that \( A \subseteq [0, M] \), and such that \( m_*(A) = 0 \). This means that for each \( \varepsilon > 0 \) there is a sequence \( \{ I_n \}_{n=1} ^{\infty}  \) of open intervals (open cubes; cf. Problem 1) such that \[A \subseteq \bigcup_{n=1}^{\infty} I_n \coloneqq \bigcup_{n=1}^{\infty} (a_n , b_n) \mbox{ and } \sum_{n=1}^{\infty}\mbox{vol}(I_n) < m_*(A) + \varepsilon  = \varepsilon .\] We claim that \( A^{2} \subseteq \bigcup_{n=1}^{\infty} I_n^{2 } = \bigcup_{n=1}^{\infty} (a_n^{2}, b_n^{2}  ) \). This holds as if \( x ^{2} \in A^{2} \implies x \in A \implies \exists \ n \in \mathbb{N}_{>0} : x \in (a_n, b_n) \implies a_n < x < b_n \implies a_n^{2} < x^{2} < b_n^{2}     \) (as all non-negative) \( \implies x^{2} \in (a^{2}_n , b^{2}_n) \subseteq \bigcup_{n=1}^{\infty} (a_n^{2} , b_n^{2} ).    \) Now, since \( A \) is bounded we let \( \ell \coloneqq \sup_{n \geq 1} \{ b_n + a_n   \} \leq 2M < \infty	 \). Thus, given \( \varepsilon > 0 \), find a covering of \( A \) by open intervals \( I_n \) such that \[ \sum_{n=1}^{\infty}\mbox{vol}(I_n) < \delta \coloneqq\frac{\varepsilon }{\sup_{n\geq 1} \{ b_n + a_n \} } > 0. \footnote{Note that if \(\sup_{n \geq 1} \{ b_n + a_n \}  = 0\), then \( A^{2} \subseteq \{ 0 \}   \) so that \( m_*(A^{2} ) = 0  \), so suppose \(\sup_{n \geq 1} \{ b_n + a_n \} > 0 \).}\] Then
\begin{align*}
	\sum_{n=1}^{\infty} \mbox{vol}(I_n^{2} )  &= \sum_{n=1}^{\infty}(b_n^{2} - a_n^{2})   \\	&= \sum_{n=1}^{\infty}(b_n - a_n)(b_n + a_n) \\
						  &\leq \ell \sum_{n=1}^{\infty}(b_n-a_n) =  \ell\sum_{n=1}^{\infty}\mbox{vol}(I_n) < \ell \delta = \varepsilon .
\end{align*}
Thus, for fixed \( \varepsilon > 0 \), we have found a covering of \( A^{2}  \) by intervals \( I_n^{2}  \) such that \(m_*(A^{2} ) <   \sum_{n=1}^{\infty} \mbox{vol}(I_n^{2} ) < \varepsilon  \). Sending \( \varepsilon \to 0 \) implies that \( m_*(A^{2} ) = 0\) for bounded \( A \). Now suppose that \( A \) is unbounded, with \( m_*(A) = 0. \) Then we can write \( A \) as \[A = \bigcup_{n=1}^{\infty} (A \cap [0, n]),\] a countable union of bounded intervals.\footnote{This holds as \( x \in A \implies \exists \ n \geq 0 : x \leq n \implies x \in A \cap [0,n] \subseteq \bigcup_{n=1}^{\infty} (A \cap [0,n] )\); and \(x \in \bigcup_{n=1}^{\infty} (A\cap [0,n]) \implies \exists \ n \geq 1 : x \in A \ \land \ x \in [0, n] \implies x \in A.\)} So that \[ m_*(A) = m_*( \bigcup_{n=1}^{\infty} (A \cap [0, n])) \leq \sum_{n=1}^{\infty}m_*(A \cap [0,n]) = \sum_{n=1}^{\infty} 0 = 0,  \] by sub-additivity and since \( A \cap [0,n] \subseteq A \) for each \( n \) implies that \( m_*(A \cap [0,n]) \leq m_*(A) = 0 \implies m_*(A \cap [0,n]) = 0.  \) We can likewise write \( A^{2} = \bigcup_{n=1}^{\infty} (A \cap [0,n])^{2} = \bigcup_{n=1}^{\infty} (A^{2} \cap [0,n^{2} ] )   \) to find that
\begin{align*}
	m_*(A^{2} )  &\leq \sum_{n=1}^{\infty} m_*(A^{2} \cap [0,n^{2}] ) \tag{applying same argument as above} \\	
		     &= \sum_{n=1}^{\infty}m_*((A \cap [0,n])^{2} ) \tag{\(\ast\)} \\
		     &= \sum_{n=1}^{\infty}m_*(A \cap [0,n]) \tag{by boundedness} \\
		     &= \sum_{n=1}^{\infty}0 = 0 \tag{By monotonicty, as argued above},
\end{align*}
where \((\ast)\) holds as \( x^{2} \in A^{2} \cap [0,n^{2} ] \implies x \in A \cap [0,n] \implies x^{2}  \in (A \cap [0,n])^{2}    \) and \( x^{2}  \in (A \cap [0,n])^{2} \implies x \in A, x \in [0,n] \implies x^{2}  \in A^{2}, x\in [0, n^{2}] , \implies x^{2} \in A^{2} \cap [0,n^{2}]       \) as needed (hence they are subsets of each other). Thus, we have proven the unbounded case as well, since \( m_*(A^{2}) \leq 0 \implies m_*(A^{2} ) = 0. \) 

Therefore, we conclude that for \( A \subseteq [0, \infty), \) \( m_*(A) = 0 \implies m_*(A^{2} ) = 0  \), thereby completing the proof.

\end{proof}
\begin{proof}[Solution For 2.2]
Let \[A \coloneqq \bigcup_{n=2}^{\infty} (n, n + \frac{1}{n^{2} } ).\] By problem 1, we use \( m_*  \) which approximates the volume of \( A \) via open cubes. Since each interval is itself an open cube, we must have that \[ \sum_{n=2}^{\infty}\mbox{vol}((n, n + \frac{1}{n^{2}}  )) = \sum_{n=2}^{\infty} \left ( {n + \frac{1}{n^{2} } - n} \right ) = \sum_{n=2}^{\infty}\frac{1}{n^{2} } \geq m_*(A).    \] By the \( p \)-test, \( \sum_{n=2}^{\infty}\frac{1}{n^{2} }  \) is a finite number which is an upper bound for \( m_*(A)  \), hence \( m_*(A) < \infty. \)

Now notice that \( A^{2} = \bigcup_{n=2}^{\infty} (n^{2}, (n + \frac{1}{n^{2} })^{2}    )  \), and by the same reasoning as above, we have \[\sum_{n=2}^{\infty}\mbox{vol}((n^{2}, (n + \frac{1}{n^{2} }  ) ^{2} ) ) = \sum_{n=2}^{\infty}(n^{2} + \frac{2n}{n^{2}} + \frac{1}{n^{4} }  - n^{2}  ) = \sum_{n=2}^{\infty}(\frac{2}{n} + \frac{1}{n^{4} } ) \geq \sum_{n=2}^{\infty} \frac{1}{n},\] since the harmonic series diverges, \(\sum_{n=2}^{\infty}\mbox{vol}((n^{2}, (n + \frac{1}{n^{2} }  ) ^{2} ) )  \) does too. But notice that \[ m_*(A^{2} ) = \sum_{n=2}^{\infty}\mbox{vol}((n^{2}, (n + \frac{1}{n^{2} }  ) ^{2} ) ) ,  \] since we have written \( A^{2}  \) (which is open as it is a union of open intervals) as a union of disjoint (cf. Lemma 2.1) open cubes (the equality thus holds by lecture). Thus, we have found a suitable example where \( m_*(A) < \infty  \) yet \( m_*(A^{2} ) = \infty  \). 
\end{proof}
\begin{proof}[Lemma 2.1]
Indeed, for fixed \( n \geq 2 \), \( (n-1+\frac{1}{(n-1)^{2} } )^{2}  = (\frac{(n-1)^{3} + 1 }{(n-1)^{2} } )^{2} = \frac{n^2(n^2-3n+3)^2}{(n-1)^{4} } \), hence \( \frac{n^2(n^2-3n+3)^2}{(n-1)^{4} } - n^{2} = \frac{n^{2}(n^{2}-3n+3 )^{2} -n^{2}(n-1)^{4}   }{(n-1)^{4} } < 0 \implies n^{2}((n^{2} - 3n + 3 )^{2} - (n-1)^{4}  ) < 0 \implies (n^{2} - 3n + 3 )^{2} - (n-1)^{4} < 0 \implies (n-2)(n^{2} + \frac{5n}{2} +2 ) < 0 \) (via tedious factoring). Thus, such holds for \( n \geq 2 \), i.e. our intervals are disjoint for \( n \geq 2 \) (as the endpoints do not intersect).
\end{proof}

\newpage
%! TeX root: ../main.tex
\noindent \textbf{Problem 3.} Let \( f : \mathbb{R}^{d} \to \mathbb{R} \) be measurable, non-negative on \( \mathbb{R}^{d}  \), and finite almost everywhere. For each \( k \in \mathbb{Z}  \), let \( F_k \coloneqq \{ x : 2^{k} < f(x) \leq 2^{k+1}  \}  \).
\begin{enumerate}
	\item Show that \( f \) is integrable on \( \mathbb{R}^{d}  \) if and only if \( \sum_{k=-\infty}^{\infty} 2^{k} m(F_k) < \infty \).
	\item Use (1) to verify that \[f(x) \coloneqq \begin{cases}
		|x|^{-a}  & |x| \leq 1 \\
		0 & \mbox{otherwise}
	\end{cases}\] is integrable over \( \mathbb{R}^{d}  \) if and only if \( a < d \).
	\item Use (1) to verify that \[g(x) \coloneqq \begin{cases}
		|x|^{-b}  & |x| > 1 \\
		0 & \mbox{otherwise}
	\end{cases}\] is integrable over \( \mathbb{R}^{d}  \) if and only if \( b > d \).
	

	
	
\end{enumerate}
Note that \( |x| = \sqrt{x_1^{2} + \cdots + x_d^{2} }  \) for all \( x \in (x_1, \hdots ,x_d) \in \mathbb{R}^{d}  \). \\

\noindent \textbf{Proof of 3.1.}

\noindent For ``$\Rightarrow$", suppose \( f \) is integrable on \( \mathbb{R}^{d}  \). Let \( \varphi \coloneqq \sum_{k=-\infty}^{\infty}2^{k} \chi_{F_{k} }  \). Then it is clear by the definition of \( F_{k}  \) that for \( k \in \mathbb{Z}  \) the \( F_{k}  \)'s are mutually disjoint.\footnote{Indeed, if \( x \in F_{i} \cap F_{j}  \) for \( i,j \in \mathbb{Z}  \) then \( x \in f ^{-1} ((2^{i} , 2^{i+1} ])  \) and \( x \in f ^{-1} ((2^{j} , 2^{j+1} ])  \). Thus \( f(x) \in (2^{i} , 2^{i+1} ] \cap (2^{j}, 2^{j+1}]   \). Suppose towards a contradiction that \( i \neq j \); without loss of generality, take \( i < j \). Then \( 2^{i+1} < 2^{i+2} \leq 2^{j}   \) implies that \( f(x) \in (2^{i} , 2^{i+1}] \cap (2^{j} , 2^{j+1} ] = \emptyset   \), a contradiction.} We note that \( F_{-\infty} = \{ x : 0 < f(x) \leq 0 \} = \emptyset  \) and \( F_{\infty} = \{ x : \infty< f(x) \leq \infty \} = \emptyset  \) as well; thus, we write \( \sum_{k=-\infty}^{\infty}2^{k} m(F_{k} ) = \sum_{k \in \mathbb{Z} }^{}2^{k} m(F_{k} ) \), since \( m(\emptyset ) = 0 \).

Let \( x \in \mathbb{R}^{d}  \) be arbitrary. If there does not exist a \( k \) for which \( x \in F_{k}  \), then \( \varphi (x) = 0 \leq f(x)\) by non-negativity. Otherwise, there is a unique \( F_{k}  \) containing \( x \) so that \( \varphi (x) = 2^{k} < f(x)  \) by definition of \( F_{k}  \). Thus, \( \varphi \leq f \) on \( \mathbb{R}^{d}  \) so that by monotonicity, \[ \int_{} \varphi \leq \int_{} f < \infty.\tag{3.1}\] Notice that by Lemma 3.1, since \( 2^{k} > 0 \) for every \( k \in \mathbb{Z}  \), \[\int_{} \varphi = \int_{} \sum_{k \in \mathbb{Z} }^{}2^{k}\chi_{F_{k} } = \sum_{k \in \mathbb{Z} }^{} \int_{} 2^{k} \chi_{F_{k} } = \sum_{k \in \mathbb{Z} }^{}2^{k} \int_{} \chi_{F_{k} } = \sum_{k \in \mathbb{Z} }^{} 2^{k} m(F_{k} ) = \sum_{k=-\infty}^{\infty}2^{k}m(F_{k}) \leq \int_{} f < \infty  ,    \] using \((3.1)\), linearity, and the definition of the integral of a characteristic function. Thus, we have that \( \sum_{k=-\infty}^{\infty} 2^{k} m(F_{k} ) < \infty \), completing the forward implication. \\

\noindent On the other hand, for \( ``$\Leftarrow$" \), suppose \( \sum_{k=-\infty}^{\infty} 2^{k} m(F_{k} ) = \sum_{k \in \mathbb{Z} }^{}2^{k} m(F_{k} ) < \infty \). Then \[2 \sum_{k=-\infty}^{\infty} 2^{k}m(F_{k}) = \sum_{k \in \mathbb{Z} }^{} 2^{k+1} m(F_{k}) < \infty.  \tag{3.2}  \] Define \( \psi \coloneqq \sum_{k \in \mathbb{Z} }^{} 2^{k+1}\chi_{F_{k} }   \). It follows for almost every \( x \in \mathbb{R}^{d}  \), \( \psi \leq f \). Indeed, for \( x \in \mathbb{R}^{d}  \), if there is no \( k \) such that \( x \in F_{k}  \), then \( f (x) = \infty \) or \(f(x) = 0 \) (such holds as \( (0,\infty) = \bigcup_{k \in \mathbb{Z} }^{} (2^{k}, 2^{k+1}] \subseteq [0,\infty],   \) which is the image of \( f \)). Since \( f \) is finite almost everywhere, we need not consider such \( x \) for which \( f(x) = \infty \); and if \( f(x) = 0 \) then \( \psi (x) = 0 = f(x) \) as needed. On the other hand, if there is an \( F_{k}  \) containing \( x \), by disjointness \( F_{k}  \) is unique so that \( \psi (x) = 2^{k+1} \geq f(x)  \) by definition of \( F_{k}  \). Thus, by monotonicity, \[\int_{}f \leq \int_{} \psi. \tag{3.3}\] Again, using Lemma 3.1, since \( 2^{k+1} > 0 \) for \( k \in \mathbb{Z}  \), we must have that \[\int_{} \psi = \int_{} \sum_{k \in \mathbb{Z} }^{} 2^{k+1} \chi_{F_{k} } = \sum_{k \in \mathbb{Z} }^{} \int_{} 2^{k+1} \chi_{F_{k} } = \sum_{k \in \mathbb{Z} }^{}2^{k+1} \int_{} \chi_{F_{k} } = \sum_{k \in \mathbb{Z} }^{}2^{k+1} m(F_{k} ) < \infty,  \] using (3.2), linearity, and the definition of the integral of a characteristic function. Thus, using \( (3.3) \), we conclude that \( f \) is integrable since \[\int_{} f \leq \int_{} \psi = \sum_{k \in \mathbb{Z} }^{}2^{k+1} m(F_{k} ) < \infty,\] thereby completing the proof.\qed \\

\noindent \textbf{Proof of 3.2.}

\noindent We first handle the trivial case where \( a \leq 0 \). Then \( a < d \) is a tautology as \( d \geq 1 \) so that \( f \) integrable \( \implies  \) \( a < d \). On the other hand, we now show that \( f \) is integrable whenever \( a \leq 0 < d \). Let \( k \coloneqq |a| \) and note that for \( x \in \overline{B(0,1)} \), \( |x| \leq 1 \implies |x|^{k-1} \leq 1 \implies |x|^{k} \leq |x| \leq 1  \) so that by monotonicity and the disjointness of a set with its complement, \[\int_{\mathbb{R}^{d} } f = \underbrace{ \int_{\mathbb{R}^{d} \setminus \overline{B(0,1)}} f}_{=0}  + \int_{\overline{B(0,1)}} f = \int_{\overline{B(0,1)}} |x|^{k} \leq \int_{\overline{B(0,1)}} 1 = \int_{\overline{B(0,1)}} \chi_{\overline{B(0,1)}} = m(\overline{B(0,1)}) < \infty   \] so that \( f \) is integrable as needed.

We now suppose that \( a > 0 \) and commence by noting that
\begin{align*}
	x \in F_{k}  &\iff f(x) \in (2^{k} , 2^{k+1} ] \iff  0 < 2^{k} < |x|^{-a} \leq 2^{k+1}  \mbox{ and } |x| \leq 1 \\
		     &\iff 0<2^{k} <|x|^{-a} \leq 2^{k+1} \mbox{ and } x \in \overline{B(0,1)}
\end{align*}
so that \[F_{k} = \{ x \in \overline{B(0,1)} : 2^{k} < |x|^{-a} \leq 2^{k+1}  \}. \] We must calculate \( m(F_{k} ) \) for each \( k \in \mathbb{Z}  \). Notice that if \( k \in \mathbb{Z} \) then any \( x \in \overline{B(0,1)} \) satisfies \( x \in F_{k}  \) if and only if
\begin{align*}
	2^{k} < |x|^{-a} \leq 2^{k+1}  &\iff 2^{k} < \left ( {\frac{1}{|x|} } \right )^{a} \leq 2^{k+1} \iff 2^{k/a} < \frac{1}{|x|} \leq 2^{\frac{k+1}{a} } \\
				       &\iff 2^{-k/a} > |x| \geq 2^{-\frac{k+1}{a} }.
\end{align*}
For \( k \geq 0 \) we have \( F_{k} = B(0, 2^{-k/a} ) \setminus B(0, 2^{-\frac{k+1}{a} } ) \). Indeed, \( B(0, 2^{-k/a} ) \setminus B(0, 2^{-\frac{k+1}{a} } ) \subseteq B(0,2^{-k/a} ) \subseteq \overline{B(0,1)}\) since \( 2^{-k/a} \leq 1  \) (the exponent is non-positive). Furthermore, for \( k \leq -1 \) we can bound \( m(F_{k}) \leq m(\overline{B(0,1)}) < \infty \) by monotonicity as \( F_{k} \subseteq \overline{B(0,1)} \).

Moreover, we claim that for \( k \geq 0 \), \( B(0,2^{-\frac{k+1}{a} }) \subseteq B(0, 2^{-k/a} )  \) since \( x \in B(0,2^{-\frac{k+1}{a} }) \implies |x| < 2^{-\frac{k+1}{a} } = \frac{1}{2^{\frac{k+1}{a} } } \leq \frac{1}{2^{\frac{k}{a} } } = 2^{-k/a}     \) so that \( x \in B(0,2^{-k/a} ) \). Since these open balls have finite radii, it follows that they have finite measure. Thus, by the excision property, \[ m(F_{k} ) = m(B(0, 2^{-k / a} ) \setminus B(0, 2^{-\frac{k+1}{a} } )) = m(B(0,2^{-k/a} ))  - m(B(0, 2^{-\frac{k+1}{a} } )). \] Now notice that \[B(0, 2^{-k / a} ) = \{ 2^{-k / a} x : x \in B(0,1)  \} \mbox{ and } B(0, 2^{-\frac{k+1}{a} } ) = \{ 2^{-\frac{k+1}{a} } x : x \in B(0,1) \}. \tag{by scaling of balls} \] Thus, using question 3 of assignment 1, we can use the dilation property of measure (with \( \delta = (2^{-k/a} , \hdots , 2^{-k/a} ) \) or \( \delta = (2^{-\frac{k+1}{a} } , \hdots , 2^{-\frac{k+1}{a} } ) \)) to obtain:
\begin{align*}
	m(F_{k}) &= m(B(0, 2^{-k  /a } )) - m(B(0, 2^{-\frac{k+1}{a} } )) = (2^{(-k/a)} )^{d} m(B(0,1)) - (2^{(-\frac{k+1}{a} )})^{d} m(B(0,1)) \\
		 &= m(B(0,1)) \cdot (2^{-dk/a} - 2^{(-dk - d)/a}    ) = m(B(0,1)) \cdot (2^{-dk/a} - 2^{-dk/a - d/a} ) \\
		 &= m(B(0,1)) \cdot (2^{-dk/a} - 2^{-dk/a}\cdot 2^{-d/a}  ) =  2^{-dk/a} m(B(0,1))(1 - 2^{-d/a} ).
\end{align*}
For the sake of clarity, let \( b_{d} \coloneqq m(B(0,1)) \) and \( \overline{b_{d} } \coloneqq m(\overline{B(0,1)}) \). For \( ``$\Rightarrow$", \) suppose \( f \) is integrable over \( \mathbb{R}^{d}  \). It follows from (1) that \[\sum_{k=-\infty}^{\infty}2^{k} m(F_{k} )  =  \sum_{k \in \mathbb{Z} }^{}2^{k} m(F_{k} ) < \infty.\] But notice that
\begin{align*}
	\sum_{k \in \mathbb{Z} }^{} 2^{k} m(F_{k} ) &= \sum_{k \in \mathbb{Z}_{\leq -1} }^{} 2^{k} m(F_{k}) + \sum_{k = 0 }^{\infty}2^{k} m(f_{k}) \\ 
						    &\leq \sum_{k \in \mathbb{Z}_{\leq -1} }^{} 2^{k}\cdot \overline{b_{d} }+ \sum_{k=0}^{\infty}2^{k} m(F_{k} ) \tag{as \( m(F_{k} ) \leq m(\overline{B(0,1)}) \)} \\
						    &= \overline{b_{d} }\cdot \sum_{k=1}^{\infty}2^{-k} +  \sum_{k=0}^{\infty}2^{k} m(F_{k} )= \overline{b_{d} } + \sum_{k=0}^{\infty}2^{k} m(F_{k} ),
\end{align*}
and since \(\sum_{k=0}^{\infty}2^{k} m(F_{k} ) \leq \sum_{k\in \mathbb{Z} _{\leq -1} }^{}2^{k} m(F_{k} ) + \sum_{k=0}^{\infty}2^{k} m(F_{k} ) = \sum_{k = -\infty}^{\infty}2^{k} m(F_{k} ) < \infty\), it follows that
\begin{align*}
	\sum_{k=0}^{\infty} 2^{k} m(F_{k} ) &= \sum_{k=0}^{\infty}2^{k} b_{d} (2^{-dk/a} )(1 - 2^{-d/a} ) = b_{d}(1-2^{-d/a} ) \sum_{k=0}^{\infty}2^{k - dk/a} \\
					    &= b_{d}(1-2^{-d/a} ) \sum_{k=0}^{\infty} 2^{k(1 - d/a)} 
\end{align*}
so that \( \sum_{k=0}^{\infty}2^{k(1 - d / a)} < \infty \). But then we must have that \( 1 - d/a  \) is negative, otherwise the sum would diverge as \(c\coloneqq 1 - d/a  \geq 0\) is fixed and \( \lim_{{k} \to {\infty}} 2^{ck} = \infty \neq 0 \). Thus \( 1 < d / a \) so that \( a < d \). 

On the other hand, for ``$\Leftarrow$", if \( a < d \) then \( \sum_{k = 0}^{\infty} 2^{k(1 - d / a)}  \) converges, since \( a < d \) means that \( \sum_{k= 0 }^{\infty} 2^{k(1 - d/a)} = \sum_{k=0}^{\infty}\frac{1}{2^{k (d / a - 1)}}  < \infty \). Then since \( c\coloneqq d/a - 1 > 0 \) is fixed, by the ratio test we have convergence: \[\lim_{{k} \to {\infty}} \left | \frac{\frac{1}{2^{ck + c} } }{\frac{1}{2^{ck} } } \right | = \frac{2^{ck} }{2^{ck} \cdot 2^{c} } = \frac{1}{2^{c} } < 1. \] Thus, \[ \sum_{k = -\infty}^{\infty}2^{k} m(F_{k} ) \leq \overline{b_{d} }+\sum_{k = 0 }^{\infty}2^{k}m(F_{k} ) < \infty   ,\] thus \( f \) is integrable by (1). Thus, the proof is complete. \qed \\

\noindent \textbf{Proof of 3.3.}
Let \( F_{k} = \{ x : 2^{k} < g(x) \leq 2^{k+1}  \}  \). We first note that if \( b \) is negative then \( g \) is unbounded on \( \mathbb{R}^{d} \setminus \overline{B(0,1)} \). This means by monotonicity that \[ \int_{\mathbb{R}^{d} \setminus \overline{B(0,1)}} g \geq \int_{\mathbb{R}^{d} \setminus \overline{B(0,1)}} \chi_{\mathbb{R}^{d} \setminus \overline{B(0,1)}}   = m(\mathbb{R}^{d} \setminus \overline{B(0,1)}) =\infty, \] since \( |x| > 1 \implies g(x) = |x|^{|b|} > 1 = \chi_{\mathbb{R}^d\setminus \overline{B(0,1)}} (x) = 1 \), for \( x \in \mathbb{R}^{d} \setminus \overline{B(0,1)} \). Thus, \( g \) is not integrable over \( \mathbb{R}^{d} \setminus \overline{B(0,1)} \) as \( \int_{\mathbb{R}^{d} \setminus \overline{B(0,1)}} g = \infty \). It follows that \( g \) can not be integrable on \( \mathbb{R}^{d} \). Hence, we may assume that \( b \geq 0 \).

Using (2), we see that for \( k \in \mathbb{Z}  \), \[ F_{k} = \{ x \in \mathbb{R}^{d} \setminus \overline{B(0,1)}  : 2^{k} < |x|^{-b} \leq 2^{k+1}  \} = \{ x \in \mathbb{R}^{d} \setminus \overline{B(0,1)}  : 2^{-k/b} > |x| \geq 2^{-\frac{k+1}{b} }  \}  .\] Thus, since \( x \in F_{k} \iff x \in B(0, 2^{-k / b} ) \setminus (\overline{B(0,1)}\cup B(0, 2^{-\frac{k+1}{b} } )) \), we have that \( F_{k} = \emptyset  \) for \( k \geq 0 \) since \( B(0, 2^{-k/b} ) \subseteq \overline{B(0,1)} \) since \( 2^{-k/b} \leq 1 \). For \( k \leq -1 \) we have \( 2^{-\frac{k+1}{b} } \geq 1 \) (since the exponent is non-negative) so that \( F_{k} = B(0, 2^{-k / b }) \setminus B(0, 2^{- \frac{k+1}{b} } ) \). Thus, using the work in (2) to calculate \( m(F_k) = 2^{-dk/b} b_{d} (1 - 2^{-d/b} )\), we obtain



\begin{align*}
	\sum_{k=-\infty}^{\infty}2^{k} m(F_{k} ) &= \sum_{k \leq - 1}^{}2^{k} m(F_{k} ) + \sum_{k \geq 0 }^{}2^{k} \cdot 0 \tag{\( k \geq 0 \implies F_{k} = \emptyset  \)} \\
						 &= b_{d} (1 - 2^{-d/b} )\sum_{k \leq - 1}^{} 2^{k(1 - d / b)} = b_{d} (1 - 2^{-d/b} )\sum_{k = 1}^{\infty} \frac{1}{2^{k(1 - d / b)}}.
\end{align*}
Now, if \( b > d \), letting \( c \coloneqq 1 - d/b > 0 \) so that \( \sum_{k=1}^{\infty} \frac{1}{2^{ck} }  \) converges by the ratio test (this was proven in (2)). Thus \( \sum_{k=-\infty}^{\infty}2^{k} m(F_{k} ) = b_{d} (1 - 2^{-d/b} )\sum_{k=1}^{\infty}2^{-k(1-d/b)} < \infty \) so that \( g \) is integrable. 


On the other hand, if \( g \) is integrable, then the above sum converges by (1). In this case, we have that \( 1 - d/b \) is positive, otherwise the sum would diverge since if \( c \coloneqq 1 - d/b < 0\) is fixed then the sum diverges since \( \lim_{{k} \to {\infty}} 1/2^{kc} = \lim_{{k} \to {\infty}}  2^{|kc|} = \infty \neq 0    \). Thus, \( 1 - d/b > 0 \iff d/b < 1 \iff b > d \). Therefore, the proof is complete.\qed

\noindent \textbf{Lemma 3.1.} Let \( \{ F_{k}  \}_{k \in \mathbb{Z} }   \) be a sequence of mutually disjoint measurable sets and \( (c_{k} )_{k \in \mathbb{Z} }  \) be a sequence of non-negative reals. Then, \[\int_{} \sum_{k \in \mathbb{Z} }^{} c_{k}\cdot \chi_{F_{k} } = \sum_{k \in \mathbb{Z} }^{} \int_{} c_{k} \chi_{F_{k} } .  \]
\begin{proof}
For \( k \in \mathbb{Z}  \) define \( a_{k}(x) \coloneqq c_{k}\cdot \chi_{F_{k} } (x)   \). Then \( a_{k}  \) is non-negative for each \( k \), and it is measurable since it is a simple function. Since \( \mathbb{Z}  \) is countable, without loss of generality, we can re-index the sequence and the sets \( F_{k} \) to both start at \( k = 1 \). Therefore, we can apply the corollary of Fatou's lemma regarding series (covered in lecture), we conclude that \[\int_{} \sum_{k\in \mathbb{Z} }^{}c_k\cdot\chi_{F_{k} } =  \int_{} \sum_{k=1}^{\infty}a_{k} = \sum_{k=1}^{\infty} \int_{} a_{k} = \sum_{k \in \mathbb{Z} }^{} \int_{} c_{k} \cdot \chi_{F_{k} } . \] 
\end{proof}

\newpage
%! TeX root: ../main.tex
\section*{Problem 4}
\noindent \textbf{4.} Let \( (A_\alpha)_{\alpha \in \mathbb{R}}  \) be a family of measurable subsets of \( \mathbb{R}^{d}  \) such that \[\bigcup_{\alpha \in \mathbb{R}}^{} A_\alpha = \mathbb{R}^{d} , \ \bigcap_{\alpha \in \mathbb{R}}^{} A_{\alpha} = \emptyset \ \mbox{and } (\forall \alpha, \beta \in \mathbb{R} : \alpha < \beta \implies A_\alpha \subseteq A_\beta).\] Find a measurable function \( f : \mathbb{R}^{d} \to \mathbb{R} \) such that \( f \leq \alpha \) on \( A_\alpha \) and \( f \geq \alpha \) on \( \mathbb{R}^{d} \setminus A_\alpha \) for each \( \alpha \in \mathbb{R} \).
\begin{proof}
	Define \( f : \mathbb{R}^{d} \to \mathbb{R} \) by \( x \xmapsto{f} \inf \{ \alpha \in \mathbb{R} : x \in A_\alpha \}  \).

	First note that since infima are unique, for each \( x \in \mathbb{R}^{d}  \) there exists a unique \( \mu \in \mathbb{R} \) such that \( f(x) = \mu \). To see why \( -\infty < \mu < \infty \), note that \( \{ \alpha \in \mathbb{R} : x \in A_\alpha \}  \) is bounded below. Indeed, since \( \bigcup_{\alpha \in \mathbb{R}}^{} A_\alpha = \mathbb{R}^{d} \), for each \( x \in \mathbb{R}^{d}  \) there exists an \( \alpha \in \mathbb{R} \) such that \( x \in A_{\alpha}  \). Then for each \( \alpha' \geq \alpha \), \( x \in A_{\alpha'}  \) since \( A_\alpha \subseteq A_{\alpha'}  \). But since \( \bigcap_{\alpha \in \mathbb{R}}^{} A_\alpha = \emptyset \), there must be a \( \lambda < \alpha \) such that \( x \notin A_{\lambda}  \), as otherwise the intersection would contain \( x \). To this end, for \( \lambda' < \lambda \), \( x \notin A_{\lambda'} \subseteq A_\lambda  \) so that \( \lambda \) is a lower bound for \( \{ \alpha \in \mathbb{R} : x \in A_\alpha \}  \). Hence, we conclude that \( \inf \{ \alpha \in \mathbb{R} : x \in A_\alpha \}  \) is a finite number, since we are taking the infimum over a bounded-below subset of \( \mathbb{R} \). 

	Let \( \alpha \in \mathbb{R} \) and \( x \in A_\alpha \) be fixed. Let \( S \coloneqq \{ \alpha \in \mathbb{R} : x \in A_\alpha \}  \) and \( \mu \coloneqq \inf S = f(x). \) By definition of the infimum, since \( \alpha \in S \), we have \( \mu \leq \alpha \), since \( \mu \) is a lower bound for \( S \). Therefore, \( f(x) = \mu \leq \alpha \). Since \( \alpha \) and \( x \) were arbitrary, we deduce that \( f \leq \alpha \) in \( A_\alpha \) for each \( \alpha \in \mathbb{R} \).

	Now suppose \( w \in \mathbb{R}^{d} \setminus A_\alpha \), with \( \rho \coloneqq f(w) = \inf \{ \alpha \in \mathbb{R} : w \in A_\alpha \} \). By definition of the infimum, for each \( \varepsilon  > 0 \) there exists a \( \rho ' \in \mathbb{R} \) such that \( w \in A_{\rho '}  \) and \( \rho' < \rho + \varepsilon \). Since \( w \in A_{\rho'}\), we must have that \( \alpha < \rho ' \), as otherwise \(w \in  A_{\rho'} \subseteq A_{\alpha}   \) so that \( w \in A_\alpha \) and \( w \in \mathbb{R}^{d} \setminus A_\alpha \), a contradiction. But then we have \( \alpha < \rho' < \rho + \varepsilon = f(w) + \varepsilon   \); and since \(\alpha, w, \mbox{and } \varepsilon  \) were arbitrary, we conclude that \( f \geq \alpha \) on \( \mathbb{R}^{d} \setminus A_\alpha \) for each \( \alpha \in \mathbb{R}\).

	It remains to be shown that \( f \) is measurable. Let \( c \in \mathbb{R} \) be fixed. Then \[ f^{-1}([-\infty, c]) =  f^{-1}((-\infty, c]) = \{ x \in \mathbb{R}^{d}  : f(x) \leq c \} = A_c \] by construction and since \( f \) only takes finite values. But \( A_c \) is measurable by hypothesis, thus \( f^{-1}([-\infty, c])  \) is measurable. Since \( c \) was arbitrary, we conclude that \( f \) is measurable with the desired properties, thereby completing the proof.
\end{proof}
%

\newpage
%! TeX root: ../main.tex
\noindent \underline{\textbf{Problem 5.}}
\begin{enumerate}
	\item Show that every closed subset of \( \mathbb{R}^{d}  \) is a \( G_\delta  \) set and every open subset of \( \mathbb{R}^{d}  \) is a \( F_\sigma  \) set.

\emph{Hint. If \( F \subseteq \mathbb{R}^{d}  \) is closed, consider \( O_n \coloneqq \{ x \in \mathbb{R}: d(x,F) < 1/n \}  \).}

\item Show that \( \mathbb{Q}  \) is an \( F_\sigma  \) set in \( \mathbb{R} \) but not a \( G_\delta  \) set.

	\emph{Hint. You may argue by contradiction: assume that \( \mathbb{Q}  \) is both an \( F_\sigma  \) set and a \( G_\delta  \), then show that there exist open sets \( (O_n)_{n \in \mathbb{N} }  \) which are all dense in \( \mathbb{R} \) and whose intersection is empty, and finally derive a contradiction with a well known property of \( \mathbb{R} \).}
\end{enumerate}
\begin{proof}[Proof of 1]$ $\newline
	\textbf{1.1. Closed subsets of \( \mathbb{R}^{d}  \) are \( G_\delta  \) sets.}

Let \( E \subseteq \mathbb{R}^{d}  \) be closed. Consider for each \( n \in \mathbb{N}_+  \) the set \( O_n \coloneqq \{ x \in \mathbb{R}^{d}  : \exists  \ p \in E : d(x,p) < 1/n  \} \). Clearly, for \( n \geq 1 \), \( E \subseteq O_n \), since if \( x \in E \), then \( d(x,x) = 0 < 1/n \). Since \( n \geq 1 \) was arbitrary, \( E \subseteq O \coloneqq \bigcap_{n=1}^{\infty} O_n. \) Now suppose \( x \in O. \) Then, for \( k = 1, 2, \hdots  \), there exists a point \( x_k \in E : d(x, x_k) < \frac{1}{k}  \). Let \( \varepsilon > 0 \) be fixed. By Archimedeanity, choose an \( N \in \mathbb{N}  \) such that \( 1/N < \varepsilon  \). Then if \( n > N \), \( d(x, x_n) < \frac{1}{n} < \frac{1}{N} < \varepsilon  \). Thus, the sequence \( (x_k)_{k \in \mathbb{N} } \subseteq E \) converges to \( x \). Since \( E \) is closed, we must have that \( x \in E \). Thus, \[E = O =  \bigcap_{n=1}^{\infty} O_n.\] It remains to be shown that \( O_n \) is open for each \( n \geq 1 \). So fix \( n \geq 1 \) and consider \( x \in O_n \). Then there exists \( p \in E : d(x,p) < \frac{1}{n}  \). Then \( x \in V_{1/n}(p) \subseteq O_n\). Indeed, if \( y \in V_{1/n}(p) \implies d(y, p) < \frac{1}{n} \implies y \in O_n.  \) Hence, we have written \( E \) as a countable intersection of open sets, hence \( E \) is a \( G_\delta  \) set.

\noindent \textbf{1.2. Open subset of \( \mathbb{R}^{d}  \) are \( F_\sigma  \) sets.}

Let \( U \subseteq \mathbb{R}^{d}  \) be open. Then \( U^{c}  \) is closed. By 1.1, we can write \( U^{c} = \bigcap_{n=1}^{\infty} O_n  \), where each \( O_n \) is some open set. But then, for \( n \geq 1 \), \( O_n^{c}  \) must be closed so that \[U = (U^{c})^{c} = \left ( {\bigcap_{n=1}^{\infty} O_n} \right ) ^{c} =   \bigcup_{n=1}^{\infty} O_n^c, \] via DeMorgan's law. Hence, we have written \( U \) as a countable union of closed sets \( O_n^{c}  \), hence \( U \) is an \( F_\sigma  \) set.\footnote{For the sake of completeness, \( U = (U^{c})^{c}   \) since \( x \in U \iff x \notin U^{c} \iff x \in (U^{c})^{c}    \).}
\end{proof}
\begin{proof}[Proof of 2]
	By lemma 5.1, \( \mathbb{Q}  \) is an \( F_\sigma  \) set in \( \mathbb{R} \). Suppose towards contradiction that \( \mathbb{Q}  \) is also a \( G_\delta  \) set in \( \mathbb{R} \). Then, there exists a sequence of open sets \( \{ \mathcal{U}_n : n \geq 1 \}  \) such that \[\mathbb{Q} = \bigcap_{n=1}^{\infty} \mathcal{U}_n.\] Clearly, for each \( n \geq 1 \), \( \mathcal{U}_n  \) is dense in \( \mathbb{R} \). Indeed, since \( \mathbb{Q}  \) is dense in \( \mathbb{R} \) and \( \mathbb{Q} = \bigcap_{n=1}^{\infty} \mathcal{U} _n \subseteq \mathcal{U}_n  \), given any \( a,b \in \mathbb{R}: a<b,\) there is a \( q \in \mathbb{Q} \subseteq \mathcal{U}_n \) such that \( a < q< b \) so that \( \mathcal{U} _n \) is dense in \( \mathbb{R} \). Since \( \mathbb{Q}  \) is countable, we can enumerate \( \mathbb{Q}  \) as a sequence \( \{ q_n : n \geq 1 \}  \).

	For each \( n \geq 1 \), define the open set \( \mathcal{O}_n \coloneqq \mathcal{U}_n \setminus \{ q_n \}  \). Then \( \mathcal{O}_n \) is still dense in \( \mathbb{R} \), since given \( a , b \in \mathbb{R} : a<b \), there were infinitely-many \( q \in \mathbb{Q}  \) satisfying \( a < q < b \), i.e. removing \( q_n \) is insignificant. Furthermore, \( \mathcal{O}_n \) is open since \( \{ q_n \}  \) being closed implies that \( \{ q_n \}^{c}  \) is open so that \( \mathcal{O}_n \cap \{ q_n \}^{c}    \) is open because finite intersections of open sets are open. Also note that \( \bigcap_{n=1}^{\infty} \mathcal{O}_n = \emptyset  \); to see why, suppose otherwise: if \( x \in \bigcap_{n = 1}^{\infty} \mathcal{O}_n \), then since for \( n \geq 1 \) \( \mathcal{O}_n = \mathcal{U}_n \setminus \{ q_n \} \subseteq \mathcal{U}_n, \ x \in \bigcap_{n=1}^{\infty} \mathcal{U}_n  = \mathbb{Q} \), a contradiction, since this means there is a \( k \geq 1 \) such that \( x = q_k \) so that \( x \notin \mathcal{O}_k = \mathcal{U}_k \setminus \{ x \} \implies x \notin \bigcap_{n=1}^{\infty} \mathcal{O}_n.    \)

	To complete the proof, we construct a sequence \( \{ F_j : j \geq 1\}   \) of compact nested intervals.
	\begin{itemize}
		\item By construction, \( \mathcal{O}_1 \neq \emptyset  \implies \exists \ x \in \mathcal{O}   _1\). By openness, there exists an \( \varepsilon_1 > 0  \) such that \( (x_1 - \varepsilon_1, x_1 + \varepsilon_1  ) \subseteq \mathcal{O}_1.  \) Thus, \( [x_1 - \frac{\varepsilon_1 }{2}, x_1 + \frac{\varepsilon_1 }{2}] \subseteq (x_1 - \varepsilon_1, x_1 + \varepsilon_1 ) \subseteq \mathcal{O}_1    \). We let \( F_1 \coloneqq  [x_1 - \frac{\varepsilon_1 }{2}, x_1 + \frac{\varepsilon_1 }{2}] \subseteq \mathcal{O} _1\).
		\item For \( k \geq 1 \), we define \( F_{k+1}  \) as follows. By the density of \( \mathcal{O}_{k+1}   \) in \( \mathbb{R} \), there exists a point \( x_{k+1} \in \mathcal{O}_{k+1}   \) such that \( x_{k+1} \in F_k^{o} .  \) Since a compact interval's interior is non-empty and open, there exists an \( \varepsilon_{k+1}' > 0  \) such that \((x_{k+1} - \varepsilon_{k+1}', x_{k+1} + \varepsilon _{k+1}' ) \subseteq F^{o}_k \subseteq F_k \). By the openness of \( \mathcal{O}_{k+1}   \), there is an \( \varepsilon_{k+1}'' > 0  \) such that \( (x_{k+1} - \varepsilon_{k+1}'', x_{k+1} + \varepsilon _{k+1}'' ) \subseteq \mathcal{O} _{k+1}  \). Letting \( \varepsilon_{k+1} \coloneqq \min \{ \varepsilon_{k+1}', \varepsilon_{k+1}''   \}   \), it follows that \( (x_{k+1} - \varepsilon_{k+1}, x_{k+1} + \varepsilon _{k+1} ) \subseteq F_k, \mathcal{O}_{k+1}    \). Thus, we define \( F_{k+1} \coloneqq [x_{k+1} - \frac{\varepsilon_{k+1}}{2}, x_{k+1} + \frac{\varepsilon _{k+1}}{2}  ] \subseteq F_k, \mathcal{O}_{k+1} .  \)
		
		
	\end{itemize}

\end{proof}
By construction, each set \( (F_j)_j \) is compact and we have \( F_{1} \supseteq F_2 \supseteq \cdots  \supseteq F_j \supseteq F_{j+1} \supseteq \cdots .   \) Thus, by the nested interval property of \( \mathbb{R} \), there exists an \( x \in \bigcap_{j=1}^{\infty} F_j \). However, by construction, \[x \in  \bigcap_{j=1}^{\infty} F_j \subseteq \bigcap_{n=1}^{\infty} \mathcal{O}_n = \emptyset, \] which holds as \( x \in \bigcap_{j=1}^{\infty} F_j \implies x \in F_j \ \forall j \geq 1 \), so that \( x \in F_1 \implies x \in \mathcal{O}_1  \), and for each \( k \geq 1 \), \( x \in F_{k} \implies x \in \mathcal{O} _{k}  \). Thus, \( x \in \bigcap_{n=1}^{\infty} \mathcal{O} _n \). But this is a contradiction as we have shown \( x \in \emptyset  \). Thus, \( \mathbb{Q}  \) is not a \( G_\delta  \) set.
\begin{proof}[Lemma 5.1]
\emph{\(\mathbb{Q}\) is an \( F_\sigma  \) set in \( \mathbb{R} \).} 

Write \(\mathbb{Q}  = \bigcup_{q \in \mathbb{Q} }^{} \{ q \}.\) Since finite sets are closed and \( \mathbb{Q}  \) is countable, \( \mathbb{Q}  \) is an \( F_\sigma  \) set in \( \mathbb{R} \) since it has been written as a countable union of closed sets. \end{proof}

%\begin{proof}[Lemma 5.1]
%\emph{Given sets \( E, A, B \subseteq \mathbb{R}^{d}  \), \( E \setminus (A \cap B) = (E \cap B) \setminus A \cup E \setminus B \).}

%Let \( x \in E \setminus (A \cap B) \) be fixed. For this to happen, we need \( x \in E, x \notin A \cap B \); this can happen in two ways, corresponding to if \( x \in B \):
%\begin{itemize}
%	\item \( x \in E, x \in B \) but \( x \notin A\), i.e. \( x \in (E \cap B) \setminus A \).
%	\item \( x \in E \) and \( x \notin B \), i.e. \( x \in E \setminus B. \)   
%\end{itemize}
%hence, we conclude that \( x \in (E \cap B)\setminus A \cup (E\setminus B) \), proving \( (\subseteq ) \). Conversely, suppose \( x \in (E \cap B) \setminus A \cup E \setminus B \), then
%\begin{itemize}
%	\item if \( x \in (E \cap B) \setminus A\), then \( x \in E \) and \( x \in B \), but \( x \notin A \implies x \notin A \cap B \). Hence \( x \in E \setminus (A \cap B) \).
%	\item if \( x \in E \setminus B \), then \( x \in E, x \notin B \implies x \notin A \cap B \implies x \in E \setminus (A \cap B) \).
%	
%\end{itemize}
%Since this covers all possible cases, we have proven \( (\supseteq) \). By definition of set equality, we are done. 
%\end{proof}
\end{document}
