\include{../preamble.tex}
\begin{document}
%! TeX root: ../main.tex
\section*{Problem 1}
\noindent \textbf{1.1.} Let \( f, g \) be continuous functions on \( (a,b) \). Show that if \( f = g \) a.e. in \( (a,b) \), then \( f = g \) in \( (a,b). \)
\begin{proof}
Let \( f, g \) be two continuous functions on \( (a,b) \) for \( a,b \in \mathbb{R}  \) with \( a < b \). Suppose \( f = g \) a.e. in \( (a,b) \). We define the set \( \mathcal{N} \coloneqq \{ x \in (a,b) : f(x) \neq g(x) \}  \) and show that it must be empty to assert the claim. Since \( f \) and \( g \) are continuous, so is their difference \( h \coloneqq f - g \). Towards a contradiction, suppose there exists \( p \in \mathcal{N}  \) so that \( f(p) \neq g(p) \). Let \( \alpha \coloneqq d(f(p) - g(p), 0) >0\). 

By the continuity of \( h \) at \( p \), given \( \varepsilon \coloneqq \alpha \) there exists a \( \delta > 0 \) such that if \( x \in (a,b) \) with \( d(x, p) < \delta  \) then \( d(h(x), h(p)) < \varepsilon . \) Since \( \mathcal{N}  \) has measure 0 there exists a point \( q \in V_\delta (p) \cap (a,b) \) such that \( q \notin \mathcal{N}  \).\footnote{This holds as otherwise, \( V_\delta (p) \cap (a,b) \) is open (finite intersection of open sets) so that for each \( q \in V_\delta (p) \cap (a,b) \) there is an \(r > 0 \) such that \(V_r(q) \subseteq V_\delta (p) \cap (a,b) \); hence \( \mathcal{N}  \) must have measure at least \( q + r - (q - r ) = 2r   > 0\), since \( V_r(q) \subseteq \mathcal{N}  \) implies by monotonicity that \(0 < 2r = m(V_r(q)) \leq m(\mathcal{N} )\), that \( m(\mathcal{N} ) > 0 \) is a contradiction} Therefore \( h(q) = f(q) - g(q) = 0. \) But then \( d(q, p) < \delta  \), yet \(d(h(q), h(p)) = d(0, f(p)-g(p)) = \alpha \not < \alpha\), a contradiction to the continuity of \( h \). Therefore, \( p \notin \mathcal{N}  \). Since \( p \) was arbitrary, we conclude that \( \mathcal{N} = \emptyset  \) and hence that \( f = g \) on \( (a,b) \).
\end{proof}

\noindent \textbf{1.2.} Show by providing a counter-example that the assertion in (1.1) is false if \( (a,b) \) is replaced by a general measurable set \( A \).
\begin{proof}[Solution]
	Let \( f : [0, 1] \cup \{ 2 \} \to \{ 5, 5.1 \}  \) be given by \[f(x) = \begin{cases}
		5 & x \in [0,1] \\
	5.1 & x = 2
\end{cases}\]
and \( g : [0,1] \cup \{ 2 \} \to \{ 5 \}  \) be given by \( g(x) = 5 \). Then \( A \coloneqq [0,1] \cup \{ 2 \}  \) is measurable (closed sets are measurable and finite unions of measurable sets are measurable). Clearly, \( f = g \) a.e. in \( [0,1] \cup \{ 2 \}  \) since they differ only on \( \{ 2 \} \) which has measure 0 since it is a finite set.
\begin{itemize}
	\item \( f \) is continuous at each \( p \in [0,1] \cup \{ 2 \}  \), since if \( p \in [0,1] \) and \( \varepsilon > 0 \) are fixed, if we let \( \delta  = \varepsilon  \) and suppose \( x \in [0,1] \cup \{ 2 \}  \) with \( d(x,p) < \delta  \), then if \( x \in [0,1] \) then \( d(f(x),f(p)) = d(5,5) = 0 < \varepsilon \), and if \( x = 2 \) with \( d(2,p) < \delta  \), then \( d(f(2), f(p)) = d(5.1, 5) = 0.1 < d(2, 1) = 1 \leq d(2, p) < \delta = \varepsilon \); likewise, if \( p = 2 \), given \( \varepsilon > 0 \) let \( \delta = \frac{1}{2}  \). Then if \( x \in [0,1] \cup \{ 2 \}  \) is such that \( d(x,2) < 1/2 \), then \( x = p \). Thus, \( d(p,p) = 0 < \delta \implies d(f(p),f(p)) = 0 < \varepsilon . \) Thus, \( f \) is continuous.
	\item \(g\) is continuous since it is uniformly continuous: given \( \varepsilon > 0 \) choosing \( \delta = \varepsilon  \) implies that for any \( x, y \in [0,1] \cup \{ 2 \}  \) with \( d(x,y) < \delta  \), we have \( d(f(x),f(y)) = d(5, 5) = 0 < \varepsilon   \).
\end{itemize}
Since \( f \) and \( g \) are both continuous and \( f = g \) a.e. on \( [0,1] \cup \{ 2 \}  \), however \( f \neq g \) on \( [0,1] \cup \{ 2 \}  \), these functions serve as a counter example to the generalised version of the claim in (1.1), as needed.
\end{proof}

\newpage
%! TeX root: ../main.tex
\noindent \underline{\textbf{Problem 2.}} Let \( A \) be a subset of \( [0, \infty) \) and \( A^{2} \coloneqq \{ x^{2} \in \mathbb{R} : x \in A \}  \).
\begin{enumerate}
	\item Prove that if \( m_*(A) = 0 \), then \( m_*(A^{2} ) = 0 \).

	\item Give an example of a set \( A \) such that \( m_*(A) < \infty \) and \( m_*(A^{2} ) = \infty \).
\end{enumerate}
\begin{proof}[Proof of 2.1]
	We first suppose that \( A \) is bounded, i.e. there is an \( M \in \mathbb{N}  \) such that \( A \subseteq [0, M] \), and such that \( m_*(A) = 0 \). This means that for each \( \varepsilon > 0 \) there is a sequence \( \{ I_n \}_{n=1} ^{\infty}  \) of open intervals (open cubes; cf. Problem 1) such that \[A \subseteq \bigcup_{n=1}^{\infty} I_n \coloneqq \bigcup_{n=1}^{\infty} (a_n , b_n) \mbox{ and } \sum_{n=1}^{\infty}\mbox{vol}(I_n) < m_*(A) + \varepsilon  = \varepsilon .\] We claim that \( A^{2} \subseteq \bigcup_{n=1}^{\infty} I_n^{2 } = \bigcup_{n=1}^{\infty} (a_n^{2}, b_n^{2}  ) \). This holds as if \( x ^{2} \in A^{2} \implies x \in A \implies \exists \ n \in \mathbb{N}_{>0} : x \in (a_n, b_n) \implies a_n < x < b_n \implies a_n^{2} < x^{2} < b_n^{2}     \) (as all non-negative) \( \implies x^{2} \in (a^{2}_n , b^{2}_n) \subseteq \bigcup_{n=1}^{\infty} (a_n^{2} , b_n^{2} ).    \) Now, since \( A \) is bounded we let \( \ell \coloneqq \sup_{n \geq 1} \{ b_n + a_n   \} \leq 2M < \infty	 \). Thus, given \( \varepsilon > 0 \), find a covering of \( A \) by open intervals \( I_n \) such that \[ \sum_{n=1}^{\infty}\mbox{vol}(I_n) < \delta \coloneqq\frac{\varepsilon }{\sup_{n\geq 1} \{ b_n + a_n \} } > 0. \footnote{Note that if \(\sup_{n \geq 1} \{ b_n + a_n \}  = 0\), then \( A^{2} \subseteq \{ 0 \}   \) so that \( m_*(A^{2} ) = 0  \), so suppose \(\sup_{n \geq 1} \{ b_n + a_n \} > 0 \).}\] Then
\begin{align*}
	\sum_{n=1}^{\infty} \mbox{vol}(I_n^{2} )  &= \sum_{n=1}^{\infty}(b_n^{2} - a_n^{2})   \\	&= \sum_{n=1}^{\infty}(b_n - a_n)(b_n + a_n) \\
						  &\leq \ell \sum_{n=1}^{\infty}(b_n-a_n) =  \ell\sum_{n=1}^{\infty}\mbox{vol}(I_n) < \ell \delta = \varepsilon .
\end{align*}
Thus, for fixed \( \varepsilon > 0 \), we have found a covering of \( A^{2}  \) by intervals \( I_n^{2}  \) such that \(m_*(A^{2} ) <   \sum_{n=1}^{\infty} \mbox{vol}(I_n^{2} ) < \varepsilon  \). Sending \( \varepsilon \to 0 \) implies that \( m_*(A^{2} ) = 0\) for bounded \( A \). Now suppose that \( A \) is unbounded, with \( m_*(A) = 0. \) Then we can write \( A \) as \[A = \bigcup_{n=1}^{\infty} (A \cap [0, n]),\] a countable union of bounded intervals.\footnote{This holds as \( x \in A \implies \exists \ n \geq 0 : x \leq n \implies x \in A \cap [0,n] \subseteq \bigcup_{n=1}^{\infty} (A \cap [0,n] )\); and \(x \in \bigcup_{n=1}^{\infty} (A\cap [0,n]) \implies \exists \ n \geq 1 : x \in A \ \land \ x \in [0, n] \implies x \in A.\)} So that \[ m_*(A) = m_*( \bigcup_{n=1}^{\infty} (A \cap [0, n])) \leq \sum_{n=1}^{\infty}m_*(A \cap [0,n]) = \sum_{n=1}^{\infty} 0 = 0,  \] by sub-additivity and since \( A \cap [0,n] \subseteq A \) for each \( n \) implies that \( m_*(A \cap [0,n]) \leq m_*(A) = 0 \implies m_*(A \cap [0,n]) = 0.  \) We can likewise write \( A^{2} = \bigcup_{n=1}^{\infty} (A \cap [0,n])^{2} = \bigcup_{n=1}^{\infty} (A^{2} \cap [0,n^{2} ] )   \) to find that
\begin{align*}
	m_*(A^{2} )  &\leq \sum_{n=1}^{\infty} m_*(A^{2} \cap [0,n^{2}] ) \tag{applying same argument as above} \\	
		     &= \sum_{n=1}^{\infty}m_*((A \cap [0,n])^{2} ) \tag{\(\ast\)} \\
		     &= \sum_{n=1}^{\infty}m_*(A \cap [0,n]) \tag{by boundedness} \\
		     &= \sum_{n=1}^{\infty}0 = 0 \tag{By monotonicty, as argued above},
\end{align*}
where \((\ast)\) holds as \( x^{2} \in A^{2} \cap [0,n^{2} ] \implies x \in A \cap [0,n] \implies x^{2}  \in (A \cap [0,n])^{2}    \) and \( x^{2}  \in (A \cap [0,n])^{2} \implies x \in A, x \in [0,n] \implies x^{2}  \in A^{2}, x\in [0, n^{2}] , \implies x^{2} \in A^{2} \cap [0,n^{2}]       \) as needed (hence they are subsets of each other). Thus, we have proven the unbounded case as well, since \( m_*(A^{2}) \leq 0 \implies m_*(A^{2} ) = 0. \) 

Therefore, we conclude that for \( A \subseteq [0, \infty), \) \( m_*(A) = 0 \implies m_*(A^{2} ) = 0  \), thereby completing the proof.

\end{proof}
\begin{proof}[Solution For 2.2]
Let \[A \coloneqq \bigcup_{n=2}^{\infty} (n, n + \frac{1}{n^{2} } ).\] By problem 1, we use \( m_*  \) which approximates the volume of \( A \) via open cubes. Since each interval is itself an open cube, we must have that \[ \sum_{n=2}^{\infty}\mbox{vol}((n, n + \frac{1}{n^{2}}  )) = \sum_{n=2}^{\infty} \left ( {n + \frac{1}{n^{2} } - n} \right ) = \sum_{n=2}^{\infty}\frac{1}{n^{2} } \geq m_*(A).    \] By the \( p \)-test, \( \sum_{n=2}^{\infty}\frac{1}{n^{2} }  \) is a finite number which is an upper bound for \( m_*(A)  \), hence \( m_*(A) < \infty. \)

Now notice that \( A^{2} = \bigcup_{n=2}^{\infty} (n^{2}, (n + \frac{1}{n^{2} })^{2}    )  \), and by the same reasoning as above, we have \[\sum_{n=2}^{\infty}\mbox{vol}((n^{2}, (n + \frac{1}{n^{2} }  ) ^{2} ) ) = \sum_{n=2}^{\infty}(n^{2} + \frac{2n}{n^{2}} + \frac{1}{n^{4} }  - n^{2}  ) = \sum_{n=2}^{\infty}(\frac{2}{n} + \frac{1}{n^{4} } ) \geq \sum_{n=2}^{\infty} \frac{1}{n},\] since the harmonic series diverges, \(\sum_{n=2}^{\infty}\mbox{vol}((n^{2}, (n + \frac{1}{n^{2} }  ) ^{2} ) )  \) does too. But notice that \[ m_*(A^{2} ) = \sum_{n=2}^{\infty}\mbox{vol}((n^{2}, (n + \frac{1}{n^{2} }  ) ^{2} ) ) ,  \] since we have written \( A^{2}  \) (which is open as it is a union of open intervals) as a union of disjoint (cf. Lemma 2.1) open cubes (the equality thus holds by lecture). Thus, we have found a suitable example where \( m_*(A) < \infty  \) yet \( m_*(A^{2} ) = \infty  \). 
\end{proof}
\begin{proof}[Lemma 2.1]
Indeed, for fixed \( n \geq 2 \), \( (n-1+\frac{1}{(n-1)^{2} } )^{2}  = (\frac{(n-1)^{3} + 1 }{(n-1)^{2} } )^{2} = \frac{n^2(n^2-3n+3)^2}{(n-1)^{4} } \), hence \( \frac{n^2(n^2-3n+3)^2}{(n-1)^{4} } - n^{2} = \frac{n^{2}(n^{2}-3n+3 )^{2} -n^{2}(n-1)^{4}   }{(n-1)^{4} } < 0 \implies n^{2}((n^{2} - 3n + 3 )^{2} - (n-1)^{4}  ) < 0 \implies (n^{2} - 3n + 3 )^{2} - (n-1)^{4} < 0 \implies (n-2)(n^{2} + \frac{5n}{2} +2 ) < 0 \) (via tedious factoring). Thus, such holds for \( n \geq 2 \), i.e. our intervals are disjoint for \( n \geq 2 \) (as the endpoints do not intersect).
\end{proof}

\newpage
%! TeX root: ../main.tex
\noindent \underline{\textbf{Problem 3.}} For every \( A \subseteq \mathbb{R}^{d}  \), \( \delta \coloneqq (\delta_1, \hdots , \delta_d) \), and \( y \coloneqq (y_1, \hdots , y_d ) \in \mathbb{R}^{d} \), define \[A_{\delta ,y} \coloneqq \{ (\delta_1 x_1 + y_1, \hdots , \delta_d x_d + y_d ) : x = (x_1, \hdots , x_d) \in A\}. \]
\underline{(1) Prove that \( m_*(A_{\delta ,y} ) = \delta_1\cdots \delta _d m_*(A). \) }
\begin{proof}[Proof of 1]
Let \( A \subseteq \mathbb{R}^{d}  \),  \( \delta \coloneqq (\delta_1, \hdots , \delta_d) \), and \( y \coloneqq (y_1, \hdots , y_d ) \in \mathbb{R}^{d} \) be arbitrary. Notice that \( A_{\delta ,y} = (\delta A) + y  \). To see this, let \( \delta x + y = (\delta_1 x_1 + y_1, \hdots , \delta_d x_d + y_d) \in A_{\delta ,y}  \). Then \( x \in (\delta A) + y\) as per its definition. The reverse inclusion likewise holds trivially: \(\delta x + y \in (\delta A) + y \implies x \in A_{\delta ,y} \). Now, we can easily apply lemmas 3.1 and 3.2 to complete the proof: \[m_*(A_{\delta ,y}) = m_*((\delta A) + y) = m_*(\delta A) = \delta_1\cdots \delta_d m_*(A).  \] Hence, \( m_*(A_{\delta ,y} ) = \delta_1\cdots \delta_d m_*(A),  \) thereby completing the proof.
\end{proof}
\noindent \underline{(2) Prove that $A$ is measurable if and only if $A_{\delta, y}$ is measurable.}
\begin{proof}$ $\newline
	\( [\implies ] \) Let \( A \subseteq \mathbb{R}^{d}  \) be measurable. We show that \( A_{\delta ,y}  \) is measurable as well, for fixed \( \delta = (\delta_1, \hdots , \delta_d) \in (0, \infty)^{d}  \) and \( y \coloneqq (y_1, \hdots , y_d) \in \mathbb{R}^{d}  \).

	Since \( A \) is measurable, for each \( \varepsilon > 0 \) there exists an open set \( \mathcal{O} _\varepsilon  \) such that \(A \subseteq  \mathcal{O} _\varepsilon  \) and \[ m_*(\mathcal{O} _\varepsilon  - A) < \frac{\varepsilon }{\delta_1\cdots \delta_d} \tag{\delta_1\cdots \delta _d > 0} . \] But notice that
\begin{align*}
	\delta_1\cdots	\delta _d \cdot m_*(\mathcal{O} _\varepsilon - A) &= m_*(\delta (\mathcal{O} _\varepsilon - A)) \tag{Lemma 3.2}\\	
					  &= m_*(\delta (\mathcal{O} _ \varepsilon - A) + y) \tag{Lemma 3.1}\\
					  &=m_*((\delta \mathcal{O} _\varepsilon +y) - (\delta A + y)) \tag {Lemma 3.4} \\
					  &= m_*(\mathcal{O}_{\varepsilon _{\delta ,y} }  - A_{\delta, y} ) < \delta_1\cdots \delta_d\cdot \frac{\varepsilon }{\delta_1 \cdots \delta_d  } = \varepsilon   \tag{by measurability of \( A \) } .
\end{align*}
By lemma 3.3, \( \mathcal{O}_{\varepsilon _{\delta ,y} }   \) is an open set, and \( A_{\delta , y} \subseteq \mathcal{O}_{\varepsilon _{\delta ,y} }     \) as if \( \delta x + y \in A_{\delta y} \implies x \in A \implies x \in \mathcal{O} _\varepsilon \implies \delta x + y \in \delta \mathcal{O} _{\varepsilon} + y \implies \delta x + y \in \mathcal{O}_{\varepsilon _{\delta ,y} }    \). Hence, since for arbitrary \( \varepsilon >0 \) we found an open set \( \mathcal{O} _{\varepsilon _{\delta ,y} }  \) such that \( A_{\delta ,y} \subseteq \mathcal{O} _{\varepsilon _{\delta ,y} }   \) and \( m_*(\mathcal{O} _{\varepsilon _{\delta ,y} } - A_{\delta ,y} ) < \varepsilon  \), we conclude that \( A_{\delta ,y}  \) is measurable. \\

\noindent \( [\impliedby ] \) For our fixed \( \delta \coloneqq (\delta_1, \hdots , \delta_d) \in (0,\infty)^{d}  \) and \( y \coloneqq (y_1, \hdots , y_d) \in \mathbb{R}^{d}  \), let \( \delta ' = (\delta_1',\hdots ,\delta_d') \coloneqq (\frac{1}{\delta_1}, \hdots ,\frac{1}{\delta _d}  ) \) (possible as for $1 \leq i \leq d,$ $\delta_i > 0$) and \( y' \coloneqq - y \). Let \( A \subseteq \mathbb{R}^{d}  \). Suppose \( A_{\delta , y}  \) is measurable; this means that for each fixed \( \varepsilon > 0 \) there exists an open set \( \mathcal{O} _\varepsilon  \) such that \( A_{\delta , y} \subseteq \mathcal{O} _\varepsilon  \) and \( m_*(\mathcal{O} _\varepsilon - A_{\delta ,y} ) < \frac{\varepsilon}{\delta_1' \cdots \delta_d'}. \) Then notice that
\begin{align}
	\delta_1'\cdots \delta_d' \cdot  m_*(\mathcal{O} _\varepsilon  - A_{\delta ,y} )  &= \delta_1'\cdots \delta_d' \cdot m_*((\mathcal{O} _\varepsilon  - A_{\delta ,y} ) + y')  \tag{Lemma 3.1} \\
							 &= \delta_1'\cdots \delta_d' \cdot m_*((\mathcal{O} _\varepsilon + y') - (A_{\delta ,y} +y')) \tag{Lemam 3.4 with \( \delta \coloneqq (1,1,\hdots ,1) \in \mathbb{R}^{d}  \) } \\
							 &= \delta_1'\cdots \delta_d' \cdot m_*((\mathcal{O}_\varepsilon  + y') - \delta A) \tag{By def. of \( A_{\delta ,y}  \) and choice of \( y' \)} \\
							 &= m_*(\delta '((\mathcal{O} _\varepsilon  + y') - \delta A)) \tag{Lemma 3.2} \\
							 &= m_*(\delta ' (\mathcal{O} _\varepsilon  + y') - A) < \delta_1'\cdots \delta_d' \cdot \frac{\varepsilon }{\delta_1'\cdots \delta_d'} = \varepsilon, \tag{$\ast$} 
\end{align}
where the last equality holds by the definition of \( \delta A  \) and choice of \( \delta ' \), and since if \( A, B \in \mathbb{R}^{d} \), \( \delta (A \setminus B) = \{ \delta x \in \mathbb{R}^{d} : x \in A, x \notin B \} = \{ \delta x  \in \mathbb{R}^{d} : \delta x \in \delta A, \delta x \notin \delta B \} =  \delta A\setminus \delta B.  \) 
It remains to be shown that \( \delta ' (\mathcal{O} _\varepsilon  + y') \) is open and contains \( A \). 
\begin{itemize}
	\item  \(\delta ' (\mathcal{O} _\varepsilon  + y')\) is open: apply lemma 3.3 to the open set \( \mathcal{O} _\varepsilon  \) with \( \delta \coloneqq (1,1, \hdots , 1) \in \mathbb{R}^{d}  \) and \( y' \) to find that \( \mathcal{O}_\varepsilon  + y' \) is open. For the sake of clarity, let \( \mathcal{U} \coloneqq \mathcal{O} _\varepsilon + y'  \). Now apply lemma 3.3 to the open set \( \mathcal{U}  \) with \( \delta ' \) and \( y \coloneqq (0,0,\hdots , 0) \in \mathbb{R}^{d}  \) to find that \( \delta ' \mathcal{U} + (0,0, \hdots , 0)  = \delta ' (\mathcal{O} _\varepsilon + y')\) is open, as needed.
	\item A \subseteq \delta ' (\mathcal{O} _\varepsilon + y')\): Let \( x = (x_1, \hdots , x_d) \in A \). Then \( (\delta_1 x_1 + y_1, \hdots , \delta_d x_d + y_d) \in A_{\delta ,y} \subseteq \mathcal{O}_\varepsilon   \). But then \( (\delta_1 x_1 + y_1 + (- y_1), \hdots , \delta_d x_d + y_d + (- y_d)) = (\delta_1 x_1 +y_1+y_1', \hdots , \delta_d x_d +y_d+y_d' ) = (\delta_1x_1, \hdots , \delta _d x_d ) \in \mathcal{O}_\varepsilon + y' \) by definition of a set's translation; but then \( x = (\frac{\delta _1}{\delta _1} x_1, \hdots ,\frac{\delta_d}{\delta_d }  x_d) = (x_1, \hdots , x_d) \in \delta ' (\mathcal{O} _\varepsilon + y')  \). Hence, \( A \subseteq  \delta ' (\mathcal{O} _\varepsilon + y')\).
\end{itemize}
Therefore, given \( \varepsilon > 0 \), we have found an open set \(  \delta ' (\mathcal{O} _\varepsilon + y') \) such that \( A \subseteq  \delta ' (\mathcal{O} _\varepsilon + y') \) and \( m_*( \delta ' (\mathcal{O} _\varepsilon + y') - A) < \varepsilon   \). Thus, \( A \) is measurable by definition, thereby completing the proof.
\end{proof}
\begin{proof}[Lemma 3.1. Translation invariance.]$ $\newline
Let \( A \subseteq \mathbb{R}^{d}  \) and \( y \in \mathbb{R}^{d}  \). Define \( A + y \coloneqq \{ x + y \in \mathbb{R}^{d} : x \in A \}. \) We will use the definition of exterior measure corresponding to coverings by open cubes (which can be done by problem 1). So suppose \( \{ C_k \}_{k=1}^{\infty}  \) is a sequence of open cubes in \( \mathbb{R}^{d}  \) such that \( A \subseteq \bigcup_{k=1}^{\infty} C_k \). Then,
\begin{align*}
	\sum_{k=1}^{\infty} \mbox{vol}(C_k)  &= \sum_{k=1}^{\infty} \mbox{vol}((a_k , b_k)^{d} ) = \sum_{k=1}^{\infty}(b_k - a_k)^{d} \\ 
					     &= \sum_{k=1}^{\infty}((b_k - a_k)^{d} + y - y) =\sum_{k=1}^{\infty}\prod_{i=1} ^{d} (b_k + y_i - (a_k + y_i)) \\ 
					     &= \sum_{k=1}^{\infty} \mbox{vol}\left ( {(a_k + y_1, b_k + y_1) \times \cdots \times (a_k + y_d, b_k + y_d)   } \right ) = \sum_{k=1}^{\infty}\mbox{vol}(C_k + y).
\end{align*}
Now notice that \( A+y \subseteq \bigcup_{k=1}^{\infty} (C_k + y) \). This holds as if \( x \in A+y \implies x - y \in A\), and since \( A \subseteq \bigcup_{k=1}^{\infty} C_k \), there exists some \( n \geq 1 \) such that \( x - y \in C_n \), but then \( x \in C_n + y \implies x \in \bigcup_{k=1}^{\infty} (C_k + y) .\) Therefore, we have shown that for any covering of \( A \subseteq \bigcup_{k=1}^{\infty} C_k \) there is a covering of \( A+y \subseteq \bigcup_{k=1}^{\infty} (C_k + y) \) such that \(\sum_{k=1}^{\infty}\mbox{vol}(C_k)  = \sum_{k=1}^{\infty}\mbox{vol}(C_k+y) \). Reading the string of equalities in the reverse order implies the exact same statement, but with the covering of \( A+y \) being fixed, and the cover of \( A \) being derived.

We now define the sets \[X \coloneqq \Set{\sum_{k=1}^{\infty} \mbox{vol}(C_k) | A \subseteq \bigcup_{k=1}^{\infty} C_k , \ C_k \mbox{ open cube} },\] and \[Y \coloneqq \Set{\sum_{k=1}^{\infty} \mbox{vol}(C_k + y) | A + y\subseteq \bigcup_{k=1}^{\infty} (C_k+y) , \ C_k + y \mbox{ open cube} }. \] By the work above, \( x \in X \implies x \in Y \) and \( x \in Y \implies x \in X \); therefore, \( X = Y \implies m_*(A) = \inf X = \inf Y = m_*(A+y).\) Therefore, the exterior measure is translation invariant. 
\end{proof}

\begin{proof}[Lemma 3.2]
	\emph{If \( A \subseteq \mathbb{R}^{d}  \) and \delta \coloneqq(\delta_1, \hdots , \delta_d ) \in (0,\infty)^{d}\), then \(m_*(\delta A) = \delta_1 \cdots \delta _d \cdot m_*(A)  \), where \( \delta A \coloneqq \{ (\delta_1 x_1, \hdots , \delta_dx_d ) : (x_1, \hdots , x_d) \in A\}  \)}. 

Let \( A \subseteq \mathbb{R}^{d}  \) and \( \delta  \coloneqq(\delta_1, \hdots , \delta_d ) \in (0,\infty)^{d} \) be fixed. Consider a covering of \( A \subseteq \bigcup_{k=1}^{\infty} C_k \) by open cubes \( C_k \) (cf. Problem 1). Then notice that
\begin{align*}
	  \prod_{i=1}^{d} \delta_i \sum_{k=1}^{\infty}\mbox{vol}((a_k, b_k)^{d} ) &=  \sum_{k=1}^{\infty}\prod_{i=1} ^{d}\delta _i \cdot (b_k - a_k)^{d} = \sum_{k=1}^{\infty} \prod_{i=1} ^{d} (\delta_i b_k - \delta_i a_k)  \\	
								      &= \sum_{k=1}^{\infty} \mbox{vol}((\delta_1a_k, \delta_1 b_k  ) \times \cdots \times (\delta_d a_k, \delta_d b_k ) )= \sum_{k=1}^{\infty}\mbox{vol}(\delta C_k) .
\end{align*}
Therefore, taking the infimum over all coverings by open cubes \( C_k \), we conclude that \[\inf \prod_{i=1}^{d} \delta_i \sum_{k=1}^{\infty}\mbox{vol}(C_k) = \prod_{i=1}^{d} \delta_i \inf \sum_{k=1}^{\infty}\mbox{vol}(C_k) = \delta_1\cdots \delta_d \cdot m_*(A) = \inf \sum_{k=1}^{\infty}\mbox{vol}(\delta C_k) = m_*(\delta A); \] indeed, for every covering by open cubes \( C_k \) of \( A \), we have \( \delta_1\cdots \delta_d \sum_{k=1}^{\infty}\mbox{vol}(C_k) = \sum_{k=1}^{\infty}\mbox{vol}(\delta C_k)  \); thus, looking at all of such possible coverings, we may deduce that the sets over which we take the infimum (when calculating the exterior measure) must be equal.

Therefore, \( m_*(\delta A) = \delta_1\cdots \delta_d \cdot m_*(A).  \)   
\end{proof}
\begin{proof}[Lemma 3.3]
\emph{For an open set \( \mathcal{O}  \subseteq \mathbb{R}^{d} \), \( \delta = (\delta_1, \hdots , \delta_d) \in (0, \infty)^{d}  \), and \( y \coloneqq (y_1, \hdots , y_d) \in \mathbb{R}^{d}  \) fixed, \( (\delta \mathcal{O} ) + y \) is open.}

To prove this lemma, suppose \( \delta x + y \in (\delta \mathcal{O} ) + y. \) Clearly, this implies that \[ x \in \mathcal{O} \implies \exists \ \varepsilon > 0 : V_\varepsilon (x) \subseteq \mathcal{O} \implies \delta V_\varepsilon (x) + y \subseteq \delta \mathcal{O} + y, \] which holds by the definition \(\delta \mathcal{O} + y \coloneqq \{ \delta x + y \in \mathbb{R}^{d} : x \in \mathcal{O}  \}\) (and note that \( \delta V_\varepsilon (x) + y \) is an open ball that has been scaled and translated, i.e. it is still an open ball). Hence, \( \delta \mathcal{O} + y \) is an open set, as \( x \) and \( \varepsilon  \) were arbitrary.
\end{proof}

\begin{proof}[Lemma 3.4]
	\emph{For an open set \( \mathcal{O}  \subseteq \mathbb{R}^{d} \), \( \delta = (\delta_1, \hdots , \delta_d) \in (O, \infty)^{d}  \), and \( y \coloneqq (y_1, \hdots , y_d) \in \mathbb{R}^{d}  \) fixed, we have \(\delta(\mathcal{O} - A) + y = (\delta \mathcal{O} + y) - (\delta A + y)\).}
	
	Suppose \( x \in \mathcal{O} - A \). Then \( \delta x + y \in \delta(\mathcal{O} - A) + y   \), and we know that \( x \notin A \), thus \( \delta x + y \notin \delta A + y \) yet \( \delta x + y \in \delta \mathcal{O} + y \); therefore, we obtain \( \delta x + y \in (\delta \mathcal{O} + y) - (\delta A + y)\). Thus \( \delta (\mathcal{O} - A) + y \subseteq (\delta \mathcal{O} + y)  - (\delta A + y). \) Conversely, if \( \delta x + y \in (\delta \mathcal{O} + y)  - (\delta A + y) \), then \( \delta x + y \in \delta \mathcal{O} + y\), but \( \delta x + y \notin \delta A + y \). But this means that \( x \notin A \implies x \in \mathcal{O} - A \implies \delta x + y \in \delta (\mathcal{O} -A) + y	\), therefore  \( \delta (\mathcal{O} - A) + y \supseteq (\delta \mathcal{O} + y) - (\delta A + y). \) By definition of set equality, the lemma is complete.
\end{proof}


\newpage
%! TeX root: ../main.tex
\noindent \underline{\textbf{Problem 4}}
\begin{enumerate}
	\item Show that a strictly increasing function that is defined on an interval has a continuous inverse.
	\item Let \( A \) and \( B \) be two Borel sets of \( \mathbb{R} \) and \( f : A \to \mathbb{R} \) be a continuous function. Show that \( f^{-1}(B)  \) is a Borel set.

		\emph{Hint. Show that the collection of sets \( B \) where \( f^{-1} (B) \) is Borel is a \(\sigma\)-algebra containing the open sets.} 
	\item Use (1) and (2) to show that a strictly increasing continuous function defined on an interval maps Borel sets to Borel sets.	
\end{enumerate}
\begin{proof}[Proof of 1]
	Let \( f : I \to \mathbb{R} \) be a strictly increasing function, where \( I \) is some interval. This means that for any \( x, y \in I \) with \( x < y \), \( f(x) < f(y). \) Define \[f^{-1}  : f(I) \to I  \mbox{ by }  f(i) \xmapsto{f^{-1} } i,\] for \( i \in I\) (this function is well-defined by injectivity, which we prove in footnote 3). We show that \( f^{-1}  \) is continuous by showing that if \( \mathcal{O} \subseteq I \) is open, then its pre-image \( (f^{-1})^{-1}  (\mathcal{O} )  \) is open. But notice that if \( f(x) \in (f^{-1})^{-1} (\mathcal{O} ) \iff f^{-1}(f(x)) \in \mathcal{O}  \iff  x \in \mathcal{O} \iff f(x) \in f(\mathcal{O} ) \), where the second to last \( \iff  \) holds since \( f \)  is injective.\footnote{\( f \) is clearly injective as if \( x, y \in I \) are such that \( x\neq y \implies x < y \) or \( x>y \); in either case, \( f(x) < f(y) \) or \( f(x) > f(y) \) respectively, so that \( f(x) \neq f(y) \), as needed. Of course, \( \mathcal{O } \subseteq f^{-1}(f(\mathcal{O} ))  \), since if \( x \in \mathcal{O} \implies f(x) \in f(\mathcal{O}) \implies  x \in f^{-1}(f(\mathcal{O} ))   \). Conversely, suppose \( x \in f^{-1}(f(\mathcal{O} )) \implies f(x) \in f(\mathcal{O} ). \) Thus, there exists an \( o \in \mathcal{O}  \) such that \( f(x) = f(o) \implies x = o \) by injectivity so that \( x \in \mathcal{O}  \) as required.} This means that \( (f^{-1})^{-1}(\mathcal{O} ) = f(\mathcal{O})   \), as these sets are subsets of each other. Thus, to show that \( f^{-1}  \) is continuous, it suffices to show that for any open subset \( \mathcal{O} \subseteq I \), \( f(\mathcal{O} ) \) is open.

	To this end, let \( \mathcal{O} \subseteq I \) be any open set, with \( x \in \mathcal{O}  \). Then there exists an \( \varepsilon > 0 \) such that \( V_\varepsilon (x) = (x - \varepsilon , x+ \varepsilon ) \subseteq \mathcal{O}  \). Then \( f(V_\varepsilon (x))  \subseteq f(\mathcal{O} ) \); but notice that for any \( y \in V_\varepsilon (x) \), \( f(y) \in f(V_\varepsilon (x)) \implies  f(x - \varepsilon ) < f(y) < f(x+ \varepsilon ) \), since \( f \) is strictly increasing. Thus, \(f(x) \in (f(x-\varepsilon ), f(x+\varepsilon )) \subseteq f(\mathcal{O} )\), an open interval, hence we conclude that \( f(\mathcal{O} ) \) is open.

	Thus, we have shown that the preimages of open sets under \( f^{-1}  \) are open. Equivalently, this means that \( f\) has a continuous inverse, as was to be shown.
\end{proof}
\begin{proof}[Proof of 2]
Let \( A, B \subseteq \mathbb{R} \) be two Borel sets and \( f: A \to \mathbb{R} \) be a continuous function. We now use lemma 4.1 to complete the proof.

By definition, the Borel \(\sigma\)-algebra is the intersection of all \(\sigma\)-algebras containing the open sets in \( \mathbb{R}^{d}  \) (\( d = 1 \) in this case). Lemma 4.1 implies that \( \Omega \), the \(\sigma\)-algebra of sets \( B \) such that \( f^{-1}(B)  \) is a Borel set is a \(\sigma\)-algebra containing the open sets. Thus, the Borel \(\sigma\)-algebra is a subset of \( \Omega \), since the Borel \(\sigma\)-algebra is the smallest \(\sigma\)-algebra containing the open sets by definition. Hence, if \( B \) is a Borel set, then \( B \in \Omega \), which means that \( f^{-1}(B) \) is a Borel set. Since \( B \) was arbitrary, the proof is complete.
\end{proof}
\begin{proof}[Proof of 3]

Let \( f : I \to \mathbb{R} \) be a strictly increasing continuous function, where \( I \) is an interval. Let \( B \subseteq I \) be any Borel set. We must show that \( f(B) \) is a Borel set.

Since \( f \) is strictly increasing, by (1) we have that \( f^{-1} : f(I) \to I  \) is continuous, where \( f^{-1}(f(i)) = i \) for \( i \in I \). Note also that since \( f \) is strictly increasing, \( f(I) \) is an interval since \( I \) is. Hence, by lemma 4.2, \( f(I) \) is a Borel set. Define \( g : f(I) \to \mathbb{R} \), where \( g(f(i)) = f^{-1}(f(i))  \) for each \( i \in I \). Then \( g \) is also continuous, since \( g(x) = f^{-1}(x)  \) for each \( x \in f(I) \) (i.e. we just extended the codomain of \( f^{-1}  \)). Thus, by (2), \( g^{-1}(B) = (f^{-1})^{-1}(B)  \) is a Borel set. But \( (f^{-1})^{-1}(B) = f(B)  \). Indeed, we showed that this is true in (1), since \( f \) is injective. Thus, \( f(B) \) is a Borel set, as needed.
\end{proof}
\begin{proof}[Lemma 4.1]
	\emph{Let \( A \subseteq \mathbb{R} \) be a Borel set and \( f: A \to \mathbb{R} \) be a continuous function. Then the collection of sets \( B \) where \( f^{-1}(B) \) is a Borel set is a \(\sigma\)-algebra containing the open sets.}

	Denote this collection of such sets \( B \) by \( \Omega. \) \( \mathbb{R} \in \Omega \) since \( f^{-1}(\mathbb{R}) = A  \) is a Borel set. Suppose \( E,F \in \Omega \); then both \( f^{-1}(E), f^{-1}(F)   \) are Borel sets so that \( f^{-1}(F) \setminus f^{-1}(E)   \) is a Borel set (since \( f^{-1}(E),f^{-1}(F) \) are in the Borel \(\sigma\)-algebra), yet \( f^{-1} (F) \setminus f^{-1}(E) = f^{-1}(F \setminus E)  \) so that \( F \setminus E \in \Omega. \)\footnote{Indeed, \( x \in f^{-1}(F) \setminus f^{-1}(E)  \iff f(x) \in F, f(x) \notin E \iff f(x) \in F\setminus E \iff x \in f^{-1}(F\setminus E) \).} Finally, if \( \{ A_k \}_{k \in \mathbb{N} } \subseteq \Omega \), then \( \{ f^{-1}(A_k)  \}_{k \in \mathbb{N} }   \) is a sequence of Borel sets so that \[ \bigcup_{k=1}^{\infty} f^{-1}(A_k) = f^{-1}\left ( {\bigcup_{k=1}^{\infty} A_k} \right )    \] is a Borel set (as each \( A_k \) belongs to the Borel \(\sigma\)-algebra) so that \( \bigcup_{k=1}^{\infty} A_k \in \Omega \).\footnote{Certainly, \( x \in \bigcup_{k=1}^{\infty} f^{-1}( A_k) \iff \exists \ m \geq 1 : x \in f^{-1}(A_m) \iff \exists \ m \geq 1 : f(x) \in A_m \iff f(x) \in \bigcup_{k=1}^{\infty} A_k \iff x \in f^{-1}(\bigcup_{k=1}^{\infty} A_k)  \).} Thus \( \Omega \) is a \(\sigma\)-algebra and it remains to be shown that if \( \mathcal{O} \subseteq \mathbb{R} \) is an open set, then \( \mathcal{O} \in \Omega \). But this is clear: by the continuity of \( f \), \( f^{-1} (\mathcal{O} ) \) is open and hence a Borel set so that \( \mathcal{O} \in \Omega \) as required, completing the lemma.
\end{proof}

\begin{proof}[Lemma 4.2]
\emph{Let \( I \subseteq \mathbb{R} \) be any interval. Then \( I \) is a Borel set.}

To prove this lemma, we use the following two facts:
\begin{enumerate}
	\item \emph{Singleton sets are Borel sets.} Let \( a \in \mathbb{R} \). Then \( \{ a \}  \) is closed so that \( \{ a \} ^{c}  \) is open and hence a Borel set. Thus, \( \mathbb{R} \setminus \{ a \} ^{c} = \{ a \}  \) is a Borel set by the difference property of the Borel \(\sigma\)-algebra.
	\item \emph{Finite unions of Borel sets are Borel sets.} This follows immediately from the countable-union property of the Borel \(\sigma\)-algebra, i.e. if \( A_1, \hdots , A_N \) are Borel sets for some \( N \in \mathbb{N}  \), then \(A_1\cup A_2 \cup \cdots \cup A_N =  \bigcup_{i=1}^{\infty} B_i\), where \( 1 \leq i \leq N \implies B_i = A_i \) and \( i > N \implies B_i = \emptyset  \) is a Borel set, since countable unions of Borel sets are Borel sets.
\end{enumerate}
Now, we have the following cases. Let \( a , b \in \mathbb{R} \) with \( a < b \).
\begin{enumerate}
	\item If \( I = \mathbb{R} \) then \( I \) is a Borel set by definition of the Borel \(\sigma\)-algebra. If \( I = (-\infty, b) \) or \( (a, \infty) \), then \( I \) is open and hence a Borel set. If \( I = (-\infty, b] \) or \( I = [a, \infty) \), then facts 1 and 2 imply that \( I = (-\infty, b) \cup \{ b \}  \) and \( I = (a, \infty) \cup \{ a \}  \) are Borel sets.
	\item \( I = [a,b] = \{ a \} \cup (a,b) \cup \{ b \} \), or \(I =  (a,b] = (a,b) \cup \{ b \} \), or \( I = [a,b) = \{ a \} \cup (a,b) \). In all of these cases, facts 1 and 2 imply that \( I \) is a Borel set.
\end{enumerate}
Since we have covered all possible cases, we conclude that \( I \) is a Borel set.
\end{proof}

\newpage
%! TeX root: ../main.tex
\noindent \underline{\textbf{Problem 5.}}
\begin{enumerate}
	\item Show that every closed subset of \( \mathbb{R}^{d}  \) is a \( G_\delta  \) set and every open subset of \( \mathbb{R}^{d}  \) is a \( F_\sigma  \) set.

\emph{Hint. If \( F \subseteq \mathbb{R}^{d}  \) is closed, consider \( O_n \coloneqq \{ x \in \mathbb{R}: d(x,F) < 1/n \}  \).}

\item Show that \( \mathbb{Q}  \) is an \( F_\sigma  \) set in \( \mathbb{R} \) but not a \( G_\delta  \) set.

	\emph{Hint. You may argue by contradiction: assume that \( \mathbb{Q}  \) is both an \( F_\sigma  \) set and a \( G_\delta  \), then show that there exist open sets \( (O_n)_{n \in \mathbb{N} }  \) which are all dense in \( \mathbb{R} \) and whose intersection is empty, and finally derive a contradiction with a well known property of \( \mathbb{R} \).}
\end{enumerate}
\begin{proof}[Proof of 1]$ $\newline
	\textbf{1.1. Closed subsets of \( \mathbb{R}^{d}  \) are \( G_\delta  \) sets.}

Let \( E \subseteq \mathbb{R}^{d}  \) be closed. Consider for each \( n \in \mathbb{N}_+  \) the set \( O_n \coloneqq \{ x \in \mathbb{R}^{d}  : \exists  \ p \in E : d(x,p) < 1/n  \} \). Clearly, for \( n \geq 1 \), \( E \subseteq O_n \), since if \( x \in E \), then \( d(x,x) = 0 < 1/n \). Since \( n \geq 1 \) was arbitrary, \( E \subseteq O \coloneqq \bigcap_{n=1}^{\infty} O_n. \) Now suppose \( x \in O. \) Then, for \( k = 1, 2, \hdots  \), there exists a point \( x_k \in E : d(x, x_k) < \frac{1}{k}  \). Let \( \varepsilon > 0 \) be fixed. By Archimedeanity, choose an \( N \in \mathbb{N}  \) such that \( 1/N < \varepsilon  \). Then if \( n > N \), \( d(x, x_n) < \frac{1}{n} < \frac{1}{N} < \varepsilon  \). Thus, the sequence \( (x_k)_{k \in \mathbb{N} } \subseteq E \) converges to \( x \). Since \( E \) is closed, we must have that \( x \in E \). Thus, \[E = O =  \bigcap_{n=1}^{\infty} O_n.\] It remains to be shown that \( O_n \) is open for each \( n \geq 1 \). So fix \( n \geq 1 \) and consider \( x \in O_n \). Then there exists \( p \in E : d(x,p) < \frac{1}{n}  \). Then \( x \in V_{1/n}(p) \subseteq O_n\). Indeed, if \( y \in V_{1/n}(p) \implies d(y, p) < \frac{1}{n} \implies y \in O_n.  \) Hence, we have written \( E \) as a countable intersection of open sets, hence \( E \) is a \( G_\delta  \) set.

\noindent \textbf{1.2. Open subset of \( \mathbb{R}^{d}  \) are \( F_\sigma  \) sets.}

Let \( U \subseteq \mathbb{R}^{d}  \) be open. Then \( U^{c}  \) is closed. By 1.1, we can write \( U^{c} = \bigcap_{n=1}^{\infty} O_n  \), where each \( O_n \) is some open set. But then, for \( n \geq 1 \), \( O_n^{c}  \) must be closed so that \[U = (U^{c})^{c} = \left ( {\bigcap_{n=1}^{\infty} O_n} \right ) ^{c} =   \bigcup_{n=1}^{\infty} O_n^c, \] via DeMorgan's law. Hence, we have written \( U \) as a countable union of closed sets \( O_n^{c}  \), hence \( U \) is an \( F_\sigma  \) set.\footnote{For the sake of completeness, \( U = (U^{c})^{c}   \) since \( x \in U \iff x \notin U^{c} \iff x \in (U^{c})^{c}    \).}
\end{proof}
\begin{proof}[Proof of 2]
	By lemma 5.1, \( \mathbb{Q}  \) is an \( F_\sigma  \) set in \( \mathbb{R} \). Suppose towards contradiction that \( \mathbb{Q}  \) is also a \( G_\delta  \) set in \( \mathbb{R} \). Then, there exists a sequence of open sets \( \{ \mathcal{U}_n : n \geq 1 \}  \) such that \[\mathbb{Q} = \bigcap_{n=1}^{\infty} \mathcal{U}_n.\] Clearly, for each \( n \geq 1 \), \( \mathcal{U}_n  \) is dense in \( \mathbb{R} \). Indeed, since \( \mathbb{Q}  \) is dense in \( \mathbb{R} \) and \( \mathbb{Q} = \bigcap_{n=1}^{\infty} \mathcal{U} _n \subseteq \mathcal{U}_n  \), given any \( a,b \in \mathbb{R}: a<b,\) there is a \( q \in \mathbb{Q} \subseteq \mathcal{U}_n \) such that \( a < q< b \) so that \( \mathcal{U} _n \) is dense in \( \mathbb{R} \). Since \( \mathbb{Q}  \) is countable, we can enumerate \( \mathbb{Q}  \) as a sequence \( \{ q_n : n \geq 1 \}  \).

	For each \( n \geq 1 \), define the open set \( \mathcal{O}_n \coloneqq \mathcal{U}_n \setminus \{ q_n \}  \). Then \( \mathcal{O}_n \) is still dense in \( \mathbb{R} \), since given \( a , b \in \mathbb{R} : a<b \), there were infinitely-many \( q \in \mathbb{Q}  \) satisfying \( a < q < b \), i.e. removing \( q_n \) is insignificant. Furthermore, \( \mathcal{O}_n \) is open since \( \{ q_n \}  \) being closed implies that \( \{ q_n \}^{c}  \) is open so that \( \mathcal{O}_n \cap \{ q_n \}^{c}    \) is open because finite intersections of open sets are open. Also note that \( \bigcap_{n=1}^{\infty} \mathcal{O}_n = \emptyset  \); to see why, suppose otherwise: if \( x \in \bigcap_{n = 1}^{\infty} \mathcal{O}_n \), then since for \( n \geq 1 \) \( \mathcal{O}_n = \mathcal{U}_n \setminus \{ q_n \} \subseteq \mathcal{U}_n, \ x \in \bigcap_{n=1}^{\infty} \mathcal{U}_n  = \mathbb{Q} \), a contradiction, since this means there is a \( k \geq 1 \) such that \( x = q_k \) so that \( x \notin \mathcal{O}_k = \mathcal{U}_k \setminus \{ x \} \implies x \notin \bigcap_{n=1}^{\infty} \mathcal{O}_n.    \)

	To complete the proof, we construct a sequence \( \{ F_j : j \geq 1\}   \) of compact nested intervals.
	\begin{itemize}
		\item By construction, \( \mathcal{O}_1 \neq \emptyset  \implies \exists \ x \in \mathcal{O}   _1\). By openness, there exists an \( \varepsilon_1 > 0  \) such that \( (x_1 - \varepsilon_1, x_1 + \varepsilon_1  ) \subseteq \mathcal{O}_1.  \) Thus, \( [x_1 - \frac{\varepsilon_1 }{2}, x_1 + \frac{\varepsilon_1 }{2}] \subseteq (x_1 - \varepsilon_1, x_1 + \varepsilon_1 ) \subseteq \mathcal{O}_1    \). We let \( F_1 \coloneqq  [x_1 - \frac{\varepsilon_1 }{2}, x_1 + \frac{\varepsilon_1 }{2}] \subseteq \mathcal{O} _1\).
		\item For \( k \geq 1 \), we define \( F_{k+1}  \) as follows. By the density of \( \mathcal{O}_{k+1}   \) in \( \mathbb{R} \), there exists a point \( x_{k+1} \in \mathcal{O}_{k+1}   \) such that \( x_{k+1} \in F_k^{o} .  \) Since a compact interval's interior is non-empty and open, there exists an \( \varepsilon_{k+1}' > 0  \) such that \((x_{k+1} - \varepsilon_{k+1}', x_{k+1} + \varepsilon _{k+1}' ) \subseteq F^{o}_k \subseteq F_k \). By the openness of \( \mathcal{O}_{k+1}   \), there is an \( \varepsilon_{k+1}'' > 0  \) such that \( (x_{k+1} - \varepsilon_{k+1}'', x_{k+1} + \varepsilon _{k+1}'' ) \subseteq \mathcal{O} _{k+1}  \). Letting \( \varepsilon_{k+1} \coloneqq \min \{ \varepsilon_{k+1}', \varepsilon_{k+1}''   \}   \), it follows that \( (x_{k+1} - \varepsilon_{k+1}, x_{k+1} + \varepsilon _{k+1} ) \subseteq F_k, \mathcal{O}_{k+1}    \). Thus, we define \( F_{k+1} \coloneqq [x_{k+1} - \frac{\varepsilon_{k+1}}{2}, x_{k+1} + \frac{\varepsilon _{k+1}}{2}  ] \subseteq F_k, \mathcal{O}_{k+1} .  \)
		
		
	\end{itemize}

\end{proof}
By construction, each set \( (F_j)_j \) is compact and we have \( F_{1} \supseteq F_2 \supseteq \cdots  \supseteq F_j \supseteq F_{j+1} \supseteq \cdots .   \) Thus, by the nested interval property of \( \mathbb{R} \), there exists an \( x \in \bigcap_{j=1}^{\infty} F_j \). However, by construction, \[x \in  \bigcap_{j=1}^{\infty} F_j \subseteq \bigcap_{n=1}^{\infty} \mathcal{O}_n = \emptyset, \] which holds as \( x \in \bigcap_{j=1}^{\infty} F_j \implies x \in F_j \ \forall j \geq 1 \), so that \( x \in F_1 \implies x \in \mathcal{O}_1  \), and for each \( k \geq 1 \), \( x \in F_{k} \implies x \in \mathcal{O} _{k}  \). Thus, \( x \in \bigcap_{n=1}^{\infty} \mathcal{O} _n \). But this is a contradiction as we have shown \( x \in \emptyset  \). Thus, \( \mathbb{Q}  \) is not a \( G_\delta  \) set.
\begin{proof}[Lemma 5.1]
\emph{\(\mathbb{Q}\) is an \( F_\sigma  \) set in \( \mathbb{R} \).} 

Write \(\mathbb{Q}  = \bigcup_{q \in \mathbb{Q} }^{} \{ q \}.\) Since finite sets are closed and \( \mathbb{Q}  \) is countable, \( \mathbb{Q}  \) is an \( F_\sigma  \) set in \( \mathbb{R} \) since it has been written as a countable union of closed sets. \end{proof}

%\begin{proof}[Lemma 5.1]
%\emph{Given sets \( E, A, B \subseteq \mathbb{R}^{d}  \), \( E \setminus (A \cap B) = (E \cap B) \setminus A \cup E \setminus B \).}

%Let \( x \in E \setminus (A \cap B) \) be fixed. For this to happen, we need \( x \in E, x \notin A \cap B \); this can happen in two ways, corresponding to if \( x \in B \):
%\begin{itemize}
%	\item \( x \in E, x \in B \) but \( x \notin A\), i.e. \( x \in (E \cap B) \setminus A \).
%	\item \( x \in E \) and \( x \notin B \), i.e. \( x \in E \setminus B. \)   
%\end{itemize}
%hence, we conclude that \( x \in (E \cap B)\setminus A \cup (E\setminus B) \), proving \( (\subseteq ) \). Conversely, suppose \( x \in (E \cap B) \setminus A \cup E \setminus B \), then
%\begin{itemize}
%	\item if \( x \in (E \cap B) \setminus A\), then \( x \in E \) and \( x \in B \), but \( x \notin A \implies x \notin A \cap B \). Hence \( x \in E \setminus (A \cap B) \).
%	\item if \( x \in E \setminus B \), then \( x \in E, x \notin B \implies x \notin A \cap B \implies x \in E \setminus (A \cap B) \).
%	
%\end{itemize}
%Since this covers all possible cases, we have proven \( (\supseteq) \). By definition of set equality, we are done. 
%\end{proof}
\end{document}
